%\include{macro_lecture}
\section*{Cíl laboratorního cvičení}
\begin{itemize}
  \item seznámit se s nástroji pro správu sítě
  \item naučit se práci s nástrojem SSH a správu klíčů 
  \item naučit se pracovat s protokolem Syslog a nástrojem rsyslog
  \item konfiguraci nástrojů pro práci s protokolem SNMP
  \item otestovat práci se syslog a SNMP
  \item seznámit se s protokolem NetFlow
  \item naučit se pracovat s nástroji {\tt nfsen} a {\tt nfdump}
\end{itemize}

\section*{Pokyny}
\begin{itemize}
  \item Do zadání nepište, slouží pro další skupiny. PDF verzi zadání
  i šablony konfiguračních souborů lze najít v IS u předmětu ISA.
  \item Na konci laboratorního cvičení nezapomeňte na poslední bod,
  tj. na {\bf Ukončení práce v laboratoři}!
\end{itemize}

\newpage
%\section{Laboratorní úlohy}
\section{Vzdálený terminál -- SSH}

Pracujte pod uživatelem {\tt user}. Zkontrolujte, že pracujete pod
uživatelem {\tt user} příkazem {\tt whoami}, který vypisuje jméno aktivního
uživatele.

Namísto řetězce {\tt <login>} používejte své studentské přihlašovací jméno.

\begin{enumerate}

  \item Ukončete démona spravujícího hesla k použitým klíčům SSH: \\ {\tt killall
    gnome-keyring-daemon}.

  \item {\bf Bezpečné připojení na vzdálený počítač bez autentizačních klíčů.}

    \begin{enumerate}

      \item Přihlaste se příkazem {\tt ssh {\em host}} na vzdálený počítač. Jako {\em host} použijte 
            jméno {\tt hNN}, kde NN je číslo stanice souseda. Uživatele není nutné specifikovat.

      \item Příkazem {\tt exit} nebo stiskem {\tt Ctrl-D} spojení ukončete.

    \end{enumerate}

  \item {\bf Vytvoření veřejného a privátního klíče.}

    \begin{enumerate}

      \item Příkazem {\tt ssh-keygen} vygenerujte implicitní klíč. Neměňte jeho
        název a zvolte heslo o délce alespoň osmi znaků.

      \item Příkazem \verb|ssh-keygen -N "" -f ~/.ssh/nopass -C <login>@nopass| vygenerujte autentizační klíč bez hesla.

      \item Příkazem {\tt ssh-keygen -N <heslo> -f \textasciitilde/.ssh/pass -C
        <login>@pass} vygeneruje klíč s heslem jiným než pro výchozí klíč.

      \item Ověřte obsah a přístupová práva u nově vzniklých souborů (ls -l \textasciitilde/.ssh). Jak
se liší práva mezi souborem s privátním a veřejným klíčem?

    \end{enumerate}

  \item {\bf Distribuce klíčů}

    \begin{enumerate}

      \item Všechny tři veřejné klíče si zkopírujte na vzdálený počítač do
        souboru \verb|.ssh/authorized_keys|.
        (např. {\verb&cat ~/.ssh/*.pub | ssh hXX "cat >> .ssh/authorized_keys"&}).

      Jaké heslo bylo nutné zadat?

      \item Zkuste se znovu přihlásit na stejný vzdálený počítač. Jaké heslo bylo nyní nutné zadat? Zkuste
zadat špatné heslo a pozorujte, které další klíče se použily. Při experimentech můžete také
využít tzv. verbose režim ssh ({\tt ssh -v}). Pro experimenty s identitou
využijte přepínač~{\tt -i}.

    \end{enumerate}

  \item {\bf Konfigurace použití klíčů}

    \begin{enumerate}

      \item Stejným způsobem nakopírujte své veřejné klíče ještě na další počítač v laboratoři.

      \item Na svém počítači vytvořte soubor \textasciitilde/.ssh/config a pomocí textového editoru přidejte
nasledující konfiguraci (hostname definujte jen pro jeden ze dvou strojů, na který jste
kopírovali veřejné klíče):
\begin{verbatim}
Host hNN hNN.netlab.fit.vutbr.cz
 IdentityFile ~/.ssh/pass

Host *
 IdentityFile ~/.ssh/id_rsa
 IdentityFile ~/.ssh/nopass
\end{verbatim}
      \item Přihlaste se postupně na oba počítače. Které klíče se použily? Co se stane, když
pro některý klíč zadáte špatné heslo?

    \end{enumerate}

  \item {\bf Omezení použití klíčů}

    Nyní bude naším cílem omezit použití klíče, který není chráněn heslem tak,
    aby pomocí něj bylo možné na vzdáleném serveru vykonat pouze konkrétní příkaz.

    \begin{enumerate}

      \item Přihlaste se na počítač, kam jste nakopírovali své veřejné klíče a
        upravte soubor s autorizovanými veřejnými klíči tak, že na začátek
        řádku s klíčem nopass (řádek poznáte tak, že končí řetězcem
        {\tt <login>@nopass}) napíšete
        \verb|command="ntpq -p" | (od původního
        obsahu oddělený jednou mezerou).

      \item Na vzdáleném počítači spusťte službu NTP příkazem {\tt systemctl
        start ntp} (jak root),

      \item Odhlaste se ze vzdáleného počítače a znovu se na něj přihlaste
        příkazem {\tt ssh hXX -i \textasciitilde/.ssh/nopass}. Aplikovalo se
        omezené využití klíče?

    \end{enumerate}


  \item {\bf Pohodlné opakované použití klíče zabezpečeného heslem.}

    \begin{enumerate}

      \item Spusťte program {\tt gnome-keyring-daemon}.

      \item Připojte se ke vzdálenému počítači za použití klíče chráněného
        heslem.

      \item Odhlaste se a znovu přihlaste. Museli jste zadávat znovu heslo?

    \end{enumerate}

      \item {\bf HTTP proxy pomocí SSH tunelu}

    \begin{enumerate}

      \item Navštivte stránku {\tt www.fit.vutbr.cz}. Všimněte si vaší IP adresy v levém dolním rohu stránky.

      \item Přihalšte se na fakultní server merlin pomocí SSH příkazu:
\begin{verbatim}
  ssh -D 8080 -N xloginNN@merlin.fit.vutbr.cz
\end{verbatim}

      \item V prohlížeči jděte v nastavení vyberte Advanced $\rightarrow$ Network $\rightarrow$
        Settings ... a nastavte adresu pro SOCKS
        Host na 127.0.0.1 a port 8080, vyberte SOCKS verze 5. Ostatní možnosti
        ponechte prázdné.

      \item Znovu navštivte stránku {\tt www.fit.vutbr.cz}. Jaká je nyní zobrazena IP adresa?

    \end{enumerate}

\end{enumerate}


\section{Syslog}
  \begin{itemize}
    \item Úkol: 
    \begin{itemize}
      \item Seznámit se s protokolem Syslog, který slouží pro přenos
      logovacích zpráv ze spravovaných zařízení. Pojmem Syslog je často označováno také
      programové vybavení implementující samotný přenos, třídění a ukládání zpráv na disk.
      \item Rozdělte se do dvojic. V každé dvojici zvolte jednu stanici jako klient a
      druhou jako server a nakonfigurujte přeposílání veškerých Syslog zpráv 
      z klienta na server. Mějte na paměti možné zneužití Syslog protokolu útočníkem a omezte
      na straně serveru příjem pouze na zprávy od daného klienta
      a klientovi zamezte příjem jakýchkoliv zpráv ze sítě.
      \item Pro práci využijte nástroj {\tt rsyslogd}, který bude sloužit jako server i klient.
      K~otestování využijte nástroj {\tt logger}.
      \item Na klientovi následně omezte přeposílání pouze na zprávy konkrétního typu.
    \end{itemize}
    \item Příkazy:
       \begin{itemize}
            \item {\tt rsyslogd(8)} -- démon pro Syslog.
            \item {\tt rsyslog.conf(5)} -- Popis konfigurace rsyslog démona.
            \item {\tt logger(1)} -- Nástroj pro generování Syslog zpráv.
            \item {\tt tcpdump(1)}
        \end{itemize}
    \item Postup:
       \begin{enumerate}
            \item Rozdělte se do dvojic a určete server a klient stanici.

            \item {\bf Na serveru} povolte naslouchání na síťovém soketu. Do souboru {\tt /etc/rsyslog.conf} přidejte:
\begin{verbatim}
  module(load="imudp")
  input(type="imudp" port="514")
\end{verbatim}

            \item {\bf Na klientovi} zakažte programu rsyslogd naslouchat na síťovém soketu,
tj. odeberte/zakomentujte v souboru {\tt /etc/rsyslogd.conf} řádky:
\begin{verbatim}
  module(load="imudp")
  input(type="imudp" port="514")

  module(load="imtcp")
  input(type="imtcp" port="514")
\end{verbatim} 

            \item {\bf Na klientovi} nakonfigurujte rsyslog démona tak, aby odesílal veškeré zprávy
         z klienta na serveri pomocí UDP. Jako oddělovač použijte výhradně tabulátor nikoliv mezery.
         Do souboru {\tt /etc/rsyslog.conf} přidejte následující pravidlo:
\begin{verbatim} 
 *.*<TAB>@<doménové_jméno_serveru>:<číslo_portu>
\end{verbatim}

            \item {\bf Na serveru i klientovi} restartujte Syslog démona: 
\begin{verbatim}
  systemctl restart rsyslog
\end{verbatim} 

            \item {\bf Z klienta} ověřte správnou konfiguraci vygenerováním testovací Syslog 
         zprávy pomocí nástroje {\tt logger}:
\begin{verbatim} 
  logger -d <obsah_zprávy>
\end{verbatim} 
        
            \item Zpráva byla přeposlána na server, kde ji lze najít na konci souboru
         {\tt /var/log/messages}.

\begin{verbatim} 
  tail -f /var/log/messages | grep <doménové_jméno_klienta>
\end{verbatim} 

            \item  Na klientovi pokračujte v generování Syslog zpráv a 
         na serveru sledujte příchozí Syslog zprávy pomocí {\tt tcpdump}. Na jakém
         portu a jakým protokolem jsou Syslog zprávy zasílány.

            \item Otevřete si manuálovou stránku rsyslog.conf a zjistěte, jaké zařízení a priority
         zpráv Syslog poskytuje. 
\begin{verbatim} 
  man rsyslog.conf
\end{verbatim} 
        
         \item Ze znalosti zařízení a priorit nakonfigurujte klienta, aby posílal na server 
         pouze zprávy při selhání autentizace, tj. upravte již existující pravidlo pro přeposílání
         veškerých zpráv na server v souboru {\tt /etc/rsyslog.conf}. Nezapomeňte restartovat Syslog démona. 
         
\begin{verbatim} 
  Syntax pravidel: <facility>.<priority><TAB><action>
\end{verbatim} 
         \item Ze serveru se následně pokuste o neúspěšné ssh připojení na klienta a podívejte se
         do souboru pro autentizační záznamy, jakou zprávu zaslal klient serveru.


       \end{enumerate}
   \end{itemize}

\section{SNMP}
  \begin{itemize}
    \item Úkol: 
    \begin{itemize}
      \item Seznámit se s  protokolem SNMP, který slouží pro přenos
      informací o stavu spravovaných zařízení (např. hodnot čítačů). Během tohoto cvičení si
      vyzkoušíte práci se synchronními SNMP událostmi, tj. model komunikace dotaz-odpověď.
      \item Každou stanici nakonfigurujte tak, aby lokálně umožňovala čtení i zápis SNMP
      proměnných a při vzdáleném přístupu k SNMP proměnným pouze čtení.
    \end{itemize}
    \item Příkazy:
       \begin{itemize}
            \item {\tt snmpd(8)} -- SNMP démon.
            \item {\tt snmpd.conf(5)} -- Popis konfigurace SNMP démona.
            \item {\tt snmpwalk(1)} -- Nástroj pro procházení a získávání podstromu hodnot SNMP
         proměnných.
            \item {\tt snmpget(1)} -- Nástroj pro získání hodnoty SNMP proměnné.
            \item {\tt snmpset(1)} -- Nástroj pro změnu hodnoty SNMP proměnné.
        \end{itemize}
    \item Postup:
       \begin{enumerate}        
            \item Otevřete si soubor {\tt /etc/snmp/snmpd.conf} pro editaci a postupně
         v něm proveďte základní konfiguraci:
           \begin{enumerate}
               \item Zvolte si vlastní název komunity - např. PUBLIC
           \item Nastavte agenta, aby poslouchal na všech lokálních adresách:
\begin{verbatim}
agentAddress udp:0.0.0.0:161
\end{verbatim}
               \item Namapujte sítě, ze kterých bude možné přistupovat k SNMP
           hodnotám. Povolte počítačům v učebně přístup pro čtení {\tt rocommunity
           <COMMUNITY> 10.10.10.0/24} a lokálnímu počítači i zápis {\tt rwcommunity <COMMUNITY>
           localhost}
           \item V konfiguračním souboru ještě nastavte SNMP proměnnou {\tt sysContact}
           na Vámi vymyšlenou hodnotu. Takto nastavená proměnná bude vždy pouze pro čtení bez ohledu
           na nastavení přístupových pravidel.
           \item Nastavte proměnnou sysLocation pomocí příkazu {\tt snmpset}
\begin{verbatim}
  snmpset -v1 -c <nazev_komunity> localhost 1.3.6.1.2.1.1.6.0 s <retezec>
\end{verbatim}
           \end{enumerate}
            \item Spusťte službu {\tt snmpd}.
\begin{verbatim}
  systemctl start snmpd
\end{verbatim} 
            \item Ověřte, že služba běží. Na kterých portech naslouchá?
\begin{verbatim}
  systemctl status snmpd
  ss -atun | grep snmpd
  ps -ax | grep snmpd | grep -v grep
\end{verbatim} 
          \item Příkazem {\tt snmpwalk} přečtěte podstrom {\tt system} a zjistěte jak váš
          soused nastavil proměnné {\tt sysLocation} a {\tt sysContact}. Např.:
\begin{verbatim}
  snmpwalk -v 1 -c <nazev_komunity> <domenove_jmeno>
  snmpwalk -v 1 -c <nazev_komunity> <domenove_jmeno> 1.3.6.1.2.1.1
  snmpwalk -v 1 -c <nazev_komunity> <domenove_jmeno> 1.3.6.1.2.1.1.6.0
  snmpwalk -v 1 -c <nazev_komunity> <domenove_jmeno> 1.3.6.1.2.1.1.4.0
\end{verbatim}

          \item Pokud znáte OID (Object Identifier) proměnné v databázi MIB (Management Information Base), pak
          můžete získat hodnotu této proměnné
          přímo pomocí {\tt snmpget}, např. počet bytů odeslaných na rozhraní eth0:
\begin{verbatim}
  snmpwalk -v 1 -c <nazev_komunity> \
           <domenove_jmeno_souseda> 1.3.6.1.2.1.2.2.1.16
  snmpget -v 1 -c <nazev_komunity> \
           <domenove_jmeno_souseda> 1.3.6.1.2.1.2.2.1.16.2
\end{verbatim}
          \item Zkuste nastavit svému sousedovi proměnnou sysLocation
          pomocí nástroje {\tt snmpset}:
\begin{verbatim}
  snmpget -v 1 -c <nazev_komunity> <domenove_jmeno_souseda> 1.3.6.1.2.1.1.6.0

  snmpset -v 1 -c <nazev_komunity> <domenove_jmeno_souseda> \
    1.3.6.1.2.1.1.6.0 s <nova_hodnota>
\end{verbatim}
          \item Proč předchozí pokus o nastavení proměnné sousedovi selhal? Zkuste nyní
          nastavit svoji proměnnou popisující název Vašeho počítače
          pomocí nástroje {\tt snmpset}:
\begin{verbatim}
  snmpget -v 1 -c <nazev_komunity> localhost 1.3.6.1.2.1.1.6.0

  snmpset -v 1 -c <nazev_komunity> localhost \
    1.3.6.1.2.1.1.6.0 s <nove_jmeno_pocitace>

  snmpget -v 1 -c <nazev_komunity> localhost 1.3.6.1.2.1.1.6.0
\end{verbatim}

\end{enumerate}
\end{itemize}

\section{NetFlow}
  \begin{itemize}
    \item Úkol:
        \begin{itemize}
            \item Seznámit se možnostmi měření provozu pomocí NetFlow. NetFlow slouží pro
            přenos statistik o jednotlivých tocích dat vznikajících při komunikaci po síti.
            Záznamy NetFlow, s nimiž budete během cvičení pracovat, jsou
            pořízeny z napojení sítě VUT a anonymizovány. V druhé části úkolu
            budete pracovat s daty pořízenými sondou FlowMon.
            \item Seznámit se s nástrojem {\tt nfsen}, který graficky zobrazuje záznamy
            NetFlow ve webovém prohlížeči. Seznámit se s nástrojem {\tt nfdump}, který slouží k dotazování na uložená data NetFlow.
        \end{itemize}
    \item Příkazy:
        \begin{itemize}
            \item {\tt nfdump}
        \end{itemize}
    \item Postup:
        \begin{enumerate}
            \item Naučte se používat nástroj {\tt nfdump}, který slouží k dotazovaní se nad záznamy NetFlow.
                \begin{enumerate}
                    \item Na Vašem počítači se v adresáři {\tt /home/user/isa5/netflow} nachází
                    anonymizovaná kolekce NetFlow dat. Tento adresář bude vstupem programu {\tt
                    nfdump}, který využjte ke kladení dotazů nad NetFlow daty.
                    \item Prostudujte manuálovou stránku nástroje {\tt nfdump}.
                    \item Dotažte se na následující statistiky. TOP 20 IP adres podle počtu přenesených bajtů. 
                        \begin{itemize}
                            \item V manuálové stránce si najděte, co dělají přepínače {\tt -R, -s, -n}.
                            \item Nezapomeňte, že zpracovávanáý data jsou relativně objemná. Dosažení výsledku tedy chvili potrvá.
                        \end{itemize}
                    \item Zjistěte, jak velké datové přenosy připadají na jednotlivé protokoly. (Statistika protokolů)
                        \begin{itemize}
                            \item Všimněte si rozdílů v podílech podle toků a podle přenesených bajtů.
                        \end{itemize}
                    \item Na základě získaných statistik se zamyslete nad velikostí sítě.
                    \item Vyfiltrujte si toky se zdrojovou IP 162.35.0.190. Zaměřte se na čísla portů.
                    Je aktivita zdroje něčím podezřelá?\footnote{Velké množství krátkých toků se
                     stejným cílem a postupně stoupajícími čísly portů ukazuje na vertikální skenování.}
                \end{enumerate}
        \item Přihlašte se na sondu FlowMon, běžící ve společnosti Flowmon Networks
            \begin{enumerate}
                \item webová stránka: {\tt https://demo.invea.cz}, login: {\tt demo}, heslo: {\tt demo}.
                      Neměňte žádné nastavení.
                \item Seznamte se s jednotlivými stránkami, které vytváří program {\tt nfsen}.
                   \begin{itemize}
                     \item {\bf Přehled} -- Stručný přehled statistik pro provoz
                     \item {\bf Zdroje} -- Zdroje NetFlow dat 
                     \item {\bf Profily} -- Nastavení pro specifický typ provozu 
                     \item {\bf Analýza} -- Podrobné informace o nasbíraných datech za určité období
                     \item {\bf Reporty} -- Vytvoření reportu za určité období
                     \item {\bf Alerty} -- Hlášení pro specifické události
                   \end{itemize}
             \item Úkoly:
                \begin{itemize}
                   \item Zjistěte, jaké IM protokoly se v této síti používají.
                   \item Zjistěte, jaké P2P protokoly se v této síti používají.
                   \item Který protokol transportní vrstvy se používá
                   nejčastěji? Jaký má přibližně podíl na celkovém provozu v
                   síti? Jak se tento podíl mění v průběhu dne, týdne? Zamyslete
                   se, proč tomu tak je.
                \end{itemize}
            \end{enumerate}       
        \end{enumerate} 
    \item Pro zájemce:
        \begin{itemize}
            \item Zájemci si mohou na stránce {\tt https://demo.invea.cz} projít
      ostatní moduly vystavěné nad protokolem. Neměňte žádné nastavení.
        \end{itemize}
    \end{itemize}
 
\section*{Ukončení práce v laboratoři}
\begin{itemize}
  \item Pod uživatelem {\tt root} spusťte dávku {\tt /root/isa5/clean} pro zrušení
  vytvořených konfiguračních souborů a pro vypnutí počítače.
\end{itemize}
