\documentclass[a4paper,11pt]{article}

\usepackage[utf8]{inputenc}
\usepackage[czech]{babel}
\usepackage[left=2cm,top=3cm,text={17cm,24cm}]{geometry}
\usepackage{graphicx}
\usepackage{listings}
\usepackage{url}
\usepackage{graphicx}
\graphicspath{ {./images/} }
\title{ISA - Laboratorní cvičení č.\,2\\
{\bf\large Zabezpečený přenos dat}}

\author{Vysoké učení technické v Brně}

\date{\url{https://github.com/nesfit/ISA/tree/master/encrypted_transfers}}

\begin{document}

{\let\newpage\relax\maketitle}

\section*{Cíl laboratorního cvičení:}
\begin{itemize}
  \item Základní seznámení se synchronizací času a protokolem NTP.
  \item Naučit se práci s nástrojem SSH a správu klíčů.
  \item Naučit se základy protokolu TLS.
  \item Seznámit se s Certificate Transparency Log.
\end{itemize}

\section*{Pokyny}
\begin{itemize}
  \item Přihlaste se do OS GNU/Linux (F3), user/password {\tt user}/{\tt user4lab}.
  \item Otevřete si příkazovou řádku pro uživatele {\tt user}.
  \item Otevřete si další příkazovou řádku pro uživatele {\tt root} příkazem {\tt su}
  (switch user).
  \item V případě potřeby si otevřete další terminál v novém okně.
  \item Pro editaci konfiguračních souborů použijte libovolný editor (např.
  nano, mcedit, vim, gedit).
\item Pracujte jednotlivě či ve dvojicích (váš počítač je dále v textu označen jako hXX a
  počítač souseda hYY, kde XX a YY je číslo počítače 01--20). Zkontrolujte, že máte počítač propojen
    přímým kabelem s přístupovým přepínačem (rozhraní eth0, zdířka E na patch
    panelu).
  
\end{itemize}

%\newpage
%\section{Laboratorní úlohy}
\section{NTP}

Vašim úkolem je zajistit synchronizaci hodin počítače. V rámci bezpečnosti je
důležité provozovat počítač se správným časem kvůli časovým razítkům
označujícím platnost a neplatnost klíčů, certifikátů, podpisů apod.

\begin{enumerate}

  \item Zkontrolujte obsah konfiguračního souboru {\tt /etc/ntp.conf}. Ujistěte
    se, že je nastavena synchronizace proti serveru {\tt ntp.fit.vutbr.cz}.

  \item Zapněte démona pro NTP ({\tt systemctl start ntp.service}).
    Sledujte postup synchronizace příkazem {\tt
    watch ntpq -p}. Nemusíte čekat na dokončení synchronizace, nechte okno
    spuštěné a pokračujte dalšími úkoly.

\end{enumerate}

\section{Vzdálený terminál -- SSH, Secure Shell}

Zkontrolujte, že máte otevřený jednu příkazovou řádku jako uživatel
{\tt user} a druhou jako uživatel {\tt root} příkazem {\tt whoami}, který vypisuje jméno aktivního
uživatele.

Namísto řetězce {\tt <login>} používejte své studentské přihlašovací jméno.

\begin{enumerate}

  \item V otevřených terminálech dočasně vypněte podporu pro SSH agenta:
  \begin{lstlisting}
  [user@hXX]$ unset SSH_AUTH_SOCK
  [root@hXX]# unset SSH_AUTH_SOCK
  \end{lstlisting}

  \item {\bf Bezpečné připojení na vzdálený počítač bez autentizačních klíčů.}

    \begin{enumerate}

      \item Spusťte program wireshark a zachytávejte komunikaci na portu 22
        (rozhraní eth0).

      \item Přihlaste se příkazem {\tt ssh user@hYY.netlab.fit.vutbr.cz.} a zadejte heslo \textbf{user4lab}.

      \item Na serveru zadejte libovolný příkaz, který znáte (např. zobrazte
        obsah manuálové stránky ssh příkazem {\tt man ssh}, vypište
        obsah adresáře příkazem {\tt ls}, obsah souborů příkazem {\tt ls},
        aktuálního uživatele příkazem {\tt whoami} apod.).

      \item Příkazem {\tt exit} nebo stiskem {\tt Ctrl-D} spojení ukončete.

      \item V programu Wireshark zobrazte obsah komunikace (follow TCP stream) a
        zjistěte informace o spojení --- verze programu {\tt ssh} na serveru i
        klientovi, podporované šifry, autentizační mechanismus HMAC, mechanismus
        pro výměnu klíčů. Ověřte, že pomocí Wiresharku nevidíte příkazy zadané v
        předchozích bodech a jejich výstup. Můžete na základě zachycené
        komunikace říct něco o jejím obsahu? Svá zjištění sdělte cvičícímu.

    \end{enumerate}

  \item {\bf Vytvoření veřejného a privátního klíče.}

    \begin{enumerate}

      \item Jako uživatel {\tt user} vygenerujte příkazem \verb|$ ssh-keygen -C <login>@user|
        implicitní klíč pro uživatele
        {\tt user}. Neměňte jeho
        název a zvolte heslo o délce alespoň osmi znaků, například
        \textbf{fitvutisa}.

      \item Jako uživatel {\tt root} vygenerujte příkazem \verb|# ssh-keygen -N "" -C <login>@root|
        implicitní klíč pro uživatele {\tt root} bez hesla.

      \item Ověřte obsah a přístupová práva u nově vzniklých souborů (ls -l \textasciitilde/.ssh). Jak
se liší práva mezi souborem s privátním a veřejným klíčem?

    \end{enumerate}

  \item {\bf Distribuce klíčů}

    \begin{enumerate}

      \item Oba veřejné klíče zkopírujte na vzdálený počítač hYY do
        souboru \verb|.ssh/authorized_keys|
        (např. {\verb&ssh-copy-id  user@hYY&} pro
        klíč uživatele {\tt user}
        a {\verb&ssh-copy-id  root@hYY&} pro
        klíč uživatele {\tt root}).
        
      Jaká hesla bylo nutné zadat?

      \item Zkuste se znovu přihlásit na stejný vzdálený počítač hYY. Jaké heslo bylo nyní nutné zadat? Zkuste
        zadat (opakovaně) špatné heslo a pozorujte, co se stalo. Při experimentech můžete také
využít tzv. verbose režim ssh ({\tt ssh -v}).

    \end{enumerate}

  \item {\bf Omezení použití klíčů}

    Nyní bude naším cílem omezit použití klíče uživatele {\tt root}, který není chráněn heslem tak,
    aby pomocí něj bylo možné na vzdáleném serveru vykonat pouze konkrétní příkaz.

    \begin{enumerate}

      \item Přihlaste se jako uživatel {\tt root} na počítač hYY, kam jste nakopírovali své veřejné klíče a
        upravte soubor s autorizovanými veřejnými klíči tak, že na začátek
        řádku s klíčem uživatele {\tt root} (řádek poznáte tak, že končí řetězcem
        {\tt <login>@root}) napíšete
        \verb|command="ntpq -p" | \\ (následovaný jednou mezerou a původním
        obsahem řádku).

      \item Odhlaste se ze vzdáleného počítače a znovu se na něj přihlaste z
        účtu {\tt root} jako {\tt root}. Aplikovalo se
        omezené využití klíče?

    \end{enumerate}


  \item {\bf Pohodlné opakované použití klíče zabezpečeného heslem.}

    \begin{enumerate}

      \item Ukončete terminál uživatele {\tt user} a vytvořte nový. Pomocí
        příkazu \verb|env| ověřte, že je nastavená proměnná
        \verb|SSH_AUTH_SOCK|.

      \item Přihlaste se k serveru hYY. Museli jste zadávat znovu heslo?

    \end{enumerate}
  \item Reportujte cvičícímu odpovědi na výše uvedené otázky.

\end{enumerate}

\section{Zabezpečení transportní vrsvy -- TLS, Transport Layer Security}

\begin{enumerate}

  \item {\bf  Nezabezpečený přenos dat}

    \begin{enumerate}

      \item Spusťte program Wireshark, zachytávejte komunikaci na portu 80.

      \item Pomocí programu {\tt telnet} se připojte k fakultnímu webovému
        serveru: \\ \verb|telnet www.fit.vutbr.cz 80|.

      \item V programu Wireshark pozorujte navázání spojení TCP pomocí
        trojcestného handshaku.

      \item Zašlete serveru požadavek protokolem HTTP, např.:
        \verb|GET / HTTP/1.0|, dotaz ukončete prázdným řádkem. V terminálu pozorujte
        odpověď.

      \item Zobrazte si v programu Wireshark komunikaci pomocí HTTP. Je možné
        přečíst obsah komunikace?

    \end{enumerate}

  \item {\bf Přenos dat zabezpečený TLS}

    \begin{enumerate}

      \item Spusťte program Wireshark, zachytávejte komunikaci na portu 443.

      \item Pomocí programu {\tt openssl} se připojte k fakultnímu webovému
        serveru: \\ \verb|openssl s_client -quiet -connect www.fit.vutbr.cz:443|.
        Všimněte si řádku \emph{Verify return code}, co
        říká o certifikátu?

      \item V programu Wireshark pozorujte navázání spojení TCP a TLS. Jaké
        informace lze vyčíst z handshaku TLS? Je možné zjistit jméno serveru?

      \item Zašlete serveru požadavek protokolem HTTP, např.:
        \verb|GET / HTTP/1.0|, dotaz ukončete prázdným řádkem. V terminálu pozorujte
        odpověď.

      \item Zobrazte si v programu Wireshark komunikaci pomocí HTTP s využitím
        TLS. Je možné přečíst obsah komunikace?

    \end{enumerate}

  \item {\bf Zabezpečený přenos dat TLS v prohlížeči}

    \begin{enumerate}

      \item Spusťte program Wireshark, zachytávejte komunikaci na portu 443.

      \item Spusťte prohlížeč Firefox a otevřete v něm stránku
        \verb|wis.fit.vutbr.cz|.

      \item Kliknutím na informace o certifikátu (vlevo od URL) zobrazte
        informace o použitém certifikátu.

      \item V menu \emph{Edit » Preferences » Privacy \& Security} v sekci
        \emph{Certificates} zobrazte důvěryhodné certifikáty. Prohlédněte si
        seznam a odhadněte jejich počet. Najděte certifikát autority Brno
        University of Technology a podívejte se na jeho detaily.

      \item V programu Wireshark pozorujte navázání spojení TCP a TLS. Jaké
        informace lze vyčíst z handshaku TLS? Je možné zjistit jméno serveru?
        Nalezené jméno serveru ukažte cvičícímu.

    \end{enumerate}

  \item {\bf Ověření použitého certifikátu}

    \begin{enumerate}

      \item Pomocí programu {\tt openssl} se připojte k WISu \\
        \verb|openssl s_client -showcerts -connect wis.fit.vutbr.cz:443|. \\
        Všimněte si řádku \emph{Verify return code}, co
        říká o certifikátu?

      \item V terminálu se přesuňte do adresáře s dočasnými soubory:
        \verb|cd /tmp| a dále pracujte v tomto adresáři. Vytvořte pracovní
        adresář a přepněte se do něj: \verb|mkdir <login> && cd <login>|.

      \item V adresáři vytvořte soubor s názvem \verb|wis.fit.vutbr.cz.pem|
        obsahující certifikát WISu. Ve výstupu příkazu
        \verb|openssl s_client -showcerts -connect wis.fit.vutbr.cz:443|
        se nachází mezi řádky
        \verb|-----BEGIN CERTIFICATE-----| a \verb|-----END CERTIFICATE-----|
        včetně těchto řádků. Použijte certifikát s kanonickým jménem (CN)
        \verb|wis.fit.vutbr.cz|.

      \item V prohlížeči se připojte na stránku \verb|http://ca.vutbr.cz/| a
        stáhněte certifikát CA ve formátu PEM, soubor stáhněte do pracovního
        adresáře. Alternaticně použijte příkaz \\
        \verb|wget http://ca.vutbr.cz/pki/pub/cacert/cacert.pem -O vut.pem|. \\
        V praxi byste tento krok měli provádět důvěryhodným kanálem.

      \item Programem {\tt c\_rehash} vytvořte symbolické odkazy, které používá
        knihovna openssl: \\ \verb|c_rehash /tmp/<login>|

      \item Znovu se připojte k WISu a sledujte řádek \emph{Verify return code},
        tentokrát příkazem: \\
        \verb|openssl s_client -showcerts -CApath /tmp/<login> -connect wis.fit.vutbr.cz:443|

      \item O svůj úspěch se podělte se svým cvičícím.

    \end{enumerate}

  \item {\bf Certificate Transparency}

    \begin{enumerate}

      \item Příkazem \verb|host -t CAA fit.vutbr.cz| si zobrazte certifikační
        autority oprávněné vydávat certifikáty pro doménu \url{fit.vutbr.cz}.

      \item V prohlížeči se připojte na stránku
        \url{https://certstream.calidog.io/}, klikněte na tačítko \emph{Open
        Fire Hose} a pozorujte nově vydávané certifikáty. K čemu si myslíte, že
        můžou být tato data dobrá? Myslíte si, že je možné data zneužít?

      \item V prohlížeči se připojte na stránku \url{https://crt.sh}.

      \item Nalezněte certifikáty vydané pro servery v rámci domény
        \emph{fit.vutbr.cz}, použijte znak \verb|%| jako zástupný znak.

      \item Najděte na stránce ikonu směřují k souboru Atom odkazující na soubor
        obsahující vystavené certifikáty. Zamyslete se k čemu byste tento soubor
        využili a ukažte tento soubor cvičícímu.

    \end{enumerate}

\end{enumerate}


\section{Ukončení práce v laboratoři}
\begin{itemize}
  \item Počítač vypněte dávkou {\tt /root/isa2/clean}.
\end{itemize}

\end{document}
%% END OF FILE 
