\setcounter{section}{0}
\section*{Teorie}
\section{Správa systémových služeb}\label{sluzby-teorie}
OS Linux poskytuje řadu různých aplikací plnících roli systémových služeb, tyto
mohou poskytovat užitečné funkce jiným aplikacím, uživatelům nebo spravovat
konfiguraci subsystémů. Systémové služby typicky běží autonomně, na pozadí
systému bez přímého rozhraní pro uživatele. Naopak systém poskytuje speciální
rozhraní pro jejich manipulaci. Pro Linuxové distribuce se \texttt{systemd} máme
dostupnou aplikaci \texttt{systemctl}, která poskytuje několik základních
příkazů pro manipulaci systémových služeb:
\begin{itemize}
    \item \texttt{systemctl start [služba]} spustí službu
    \item \texttt{systemctl stop [služba]} zastaví službu
    \item \texttt{systemctl restart [služba]} restartuje službu
    \item \texttt{systemctl enable [služba]} povolí automatické spuštění služby
        po startu systému
    \item \texttt{systemctl disable [služba]} zakáže automatické spuštění
        služby po startu systému
    \item \texttt{systemctl status [služba]} vypíše informace aktuálního stavu
        služby a několik posledních řádků systémových logů této služby. Tento
        příkaz můžete použít i bez specifikace služby, dostanete tak informace
        o všech aktuálně spuštěných službách.
\end{itemize}

Jméno služby má typicky tvar \texttt{<aplikace>.service}, například
\texttt{sshd.service}.

Pro zobrazení kompletních logů konkrétní systémové služby můžete použít příkaz
\texttt{journalctl -u [služba]}.

Detailní popis včetně možných voleb a příkladů použití můžete nalézt
v~manuálových stránkách.

\section{Adresy v IPv4 sítí}\label{ipv4-teorie}

\subsection{Formát adresy}
IPv4 adresa je délky 32 bitů a preferovaný zápis má formát {\tt X.X.X.X}, kde
{\tt X} je decimální zápis 8 bitového čísla. Příklad:\\
\begin{verbatim}
8.8.8.8
127.0.0.1
\end{verbatim}

Adresa dvě části:
\begin{itemize}
    \item adresa sítě ({\bf prefix}) a
    \item adresa uzlu.
\end{itemize}
Délka prefixu se zapisuje v desítkovém tvaru za lomítko. {\tt 192.168.0.1/24}

V síti s daným prefixem existují 2 speciální adresy, které nelze použít pro
adresování jednotlivých uzlů:
\begin{itemize}
    \item adresa sítě - adresa uzlu je nulová, např. 192.168.0.0/24
    \item broadcastová adresa - nejvyšší možná adresa uzlu pro daný prefix,
        např. 192.168.0.255/24
\end{itemize}

\section{Adresy v IPv6 sítí}\label{ipv6-teorie}

\subsection{Formát adresy}
IPv6 adresa je délky 128 bitů a zapsaných ve formátu {\tt X:X:X:X:X:X:X:X}, kde
{\tt X} je hexadecimální zápis 16 bitového čísla.
Příklad:\\
\begin{verbatim}
FEDC:BA98:7654:3210:fedc:ba98:7654:3210
1080:0000:0000:0000:0008:0800:200C:417a
\end{verbatim}

Preferovaný formát je dále upraven v RFC 5952\footnote{https://tools.ietf.org/html/rfc5952}, které definuje následující
pravidla:
\begin{itemize}
    \item Nuly na začátku každého 16 bitového čísla je potřeba vynechat (např.
        80 namísto 0080 a 0 namísto 0000).
    \item Více nulových bloků lze nejvýše jednou nahradit znakem {\tt ::}, a
        MUSÍ být nahrazena nejdelší možná posloupnost takových nulových bloků.
        V případě více shodně dlouhých posloupností, nahrazuje se ta nejvíce
        vlevo.
    \item Znak {\tt ::} nesmí nahrazovat samostatný nulový blok.
    \item Znaky "a", "b", "c", "d", "e", "f" hexadecimální soustavy se vždy
        píšou malými písmeny.
\end{itemize}
Adresy z předchozího příkladu tedy budou vypadat následovně:\\
\begin{verbatim}
fedc:ba98:7654:3210:fedc:ba98:7654:3210
1080::8:800:200c:417a
\end{verbatim}

Stejně jako v IPv4 má adresa dvě části: \emph{adresa sítě ({\bf prefix})} a
\emph{adresa uzlu}. Délká prefixu se zapisuje v desítkovém tvaru za lomítko
(např. {\tt 1080::/60}).

\subsection{Rozdělení adres}
IPv6 adresy je možno rozdělit podle rozsahu. Typicky může jít o tři
možnosti -- adresy na lince (neprojdou za router), lokální adresy (ULA, nejsou
routovatelné ve veřejné sítí) a veřejné adresy.

\begin{table}[ht!]
    \begin{center}
        \begin{tabular}{l|l|l}
            Význam & Prefix (bitově) & Prefix \\
            \hline\hline
            Veřejné  & {\tt 001} & {\tt 2000::/3} \\
            \hline
            Lokální  & {\tt 1111 110} & {\tt FC00::/7} \\
            \hline
            Linkové  & {\tt 1111 1110 10} & {\tt FE80::/10} \\
            \hline
            Multicast & {\tt 1111 1111} & {\tt FF00::/8} \\
            \hline
        \end{tabular}
    \end{center}
\end{table}

\subsection{Lokální IPv6 (ULA) adresy}\label{ula}
Síťová část ULA adresy se stává ze 4 části:
\begin{description}
    \item [Prefix] {\tt fc00::/7}
    \item [L] bit 1 pokud byl prefix přiřazen lokálně, 0 zatím nebyla
        definována, začátek adresy je proto typicky {\tt fd00::/8}
    \item [Global ID] identifikátor sítě, měl by být unikátní, standard
        popisuje pseudonáhodný algoritmus pro generování. Unikátnost je
        požadována, aby při spojení více lokálních sítí nebylo třeba žádnou
        přečíslovat.
    \item [Inderface ID] Identifikátor rozhraní, existuje několik variant jak
        jej získat. Původní standard (RFC
        3513\footnote{https://tools.ietf.org/html/rfc3513} a RFC
        4291\footnote{https://tools.ietf.org/html/rfc4291}) doporučoval použití
        EUI-64. V současnosti byla tato metoda nahrazena RFC
        7217\footnote{https://tools.ietf.org/html/rfc7217}. Pro ochranu
        soukromí jsou další specifika automatické generace identifikátorů
        definovány v RFC 4941\footnote{https://tools.ietf.org/html/rfc4941}.
\end{description}

\begin{table}[ht!]
    \begin{center}
        \begin{tabular}{c|c|c|c|c}
            7 bits & 1 &  40 bits  &  16 bits  & 64 bits \\
            \hline
            1111 110 & L & Global ID & Subnet ID & Interface ID \\
            \hline
        \end{tabular}
    \end{center}
\end{table}

Lokální síť je tedy složená ze tří části:
\begin{itemize}
    \item unikátní prefix délky 48 bitů,
    \item 16 bitový identifikátor podsítě,
    \item adresa uzlu o délce 64 bitů.
\end{itemize}

\subsection{Manuální konfigurace IP Adres}\label{ip-manual}
Pro dočasnou konfiguraci IP adres na OS Linux slouží příkaz {\tt ip}. Pro
dnešní cvičení budeme potřebovat následující příkazy:
\begin{itemize}
    \item \verb_ip link set [rozhraní] up/down_ pro zapnutí/vypnutí rozhraní
    \item \verb_ip addr add/del [IP adresa]/[prefix] dev [rozhraní]_ pro
        přidání/odstranění adresy z rozhraní
    \item \verb_ip addr flush dev [rozhraní]_ pro odstranění všech adres z
        rozhraní
    \item \verb_ip route add default via [IP adresa]_ pro nastavení výchozí
        brány
    \item \verb_ip link_, \verb_ip addr_, \verb_ip route_ zobrazí aktuální
        konfiguraci
\end{itemize}

\subsection{Dynamická konfigurace IPv4 - DHCP}\label{dhcp}
Pro dynamickou konfiguraci IP adres je potřeba aby byl na síti přístupný
nakonfigurovaný DHCP server a klienti, kteří si o IP adresu požádají. Kromě
přiřazení IP adres má DHCP server na starosti i šíření jiných informací
důležitých pro bezproblémovou funkčnost sítě. Jednou takovou informací je IP
adresa doporučeného DNS serveru pro klienty na síti.

Na Vašich počítačích máte nainstalovanou serverovou aplikaci ISC DHCP, která
poskytuje služby DHCP serveru. Aplikace se konfiguruje pomocí souboru
\verb_/etc/dhcp/dhcpd.conf_.

Příklad obsahu konfiguračního souboru:
\begin{verbatim}
option domain-name-servers [IP adresy DNS serverů];
subnet [IP adresa sítě] netmask [maska] {
    range [první přiřaditelná IP adresa] [poslední přiřaditelná IP adresa];
}
\end{verbatim}
Dodatečné informace o konfiguračních možnostech najdete v manuálových stránkách
{\tt man dhcpd.conf} případně {\tt man dhcpd}.

Službu DHCP serveru možno spustit pomocí příkazu:
\begin{verbatim}
systemctl start isc-dhcp-server.service
\end{verbatim}

DHCP klient je implementován v aplikaci \texttt{dhclient}. Může být spuštěn bez
parametrů pro všechna aktivní síťová rozhraní nebo lépe s jménem konkrétního
síťového rozhraní, které má být konfigurováno:
\begin{verbatim}
dhclient -v [rozhraní]
\end{verbatim}
Argument \texttt{-v} zajistí, že aplikace vypíše detailní informace.
Chybu týkající se {\tt smbd.service} někdy zobrazovanou v laboratoři můžete ignorovat.

\subsection{Dynamická konfigurace IPv6}\label{dyn-ipv6}
Dynamická konfigurace IPv6 se dělí na {\bf bezstavovou} a {\bf stavovou}
konfiguraci. Na cvičeních se budeme zabývat jen bezstavovou. Bezstavová
konfigurace byla navržena tak, aby stačilo připojit zařízení do sítě a
automaticky si klient vygeneroval nějakou adresu a ihned mohl komunikovat se
světem. K tomu klient potřebuje znát prefix sítě do které byl připojen.
K~tomuto účelu se používají zprávy označované jako Routing Advertisement (RA).
Tyto a ještě další zprávy jsou součásti procesu Neighbor Discovery (RFC
4861\footnote{https://tools.ietf.org/html/rfc4861}).

Na počítačích v laboratoři je nainstalovaná systémová služba {\tt radvd.service},
která je schopna vysílat zprávy RA. Aplikaci je možno nakonfigurovat pomocí
souboru {\tt /etc/radvd.conf}.

Příklad konfigurace:
\begin{verbatim}
interface [rozhraní]
{
    AdvSendAdvert on;
    MaxRtrAdvInterval [Max pocet sekund mezi zpravami RA, min 4];
    prefix [prefix]/[delka prefixu]
    {
        AdvOnLink on;
        AdvAutonomous on;
        AdvRouterAddr on;
    };
};
\end{verbatim}
Přehled všech možností konfigurace poskytnou manualové stránky {\tt man radvd}
a {\tt man radvd.conf}.

Pro správné šíření RA zpráv je potřebné povolit směrování IPv6 provozu:
\begin{verbatim}
sysctl net.ipv6.conf.all.forwarding=1
\end{verbatim}

Součásti balíčku radvd je i aplikace {\tt radvdump}, kterou možno použít pro
analýzu -- naslouchá na síťových rozhraních a tiskne na obrazovku obsah
zachycených RA zpráv.
