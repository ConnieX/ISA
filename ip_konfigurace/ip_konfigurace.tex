\documentclass[a4paper,11pt,draft]{article}

\usepackage[utf8]{inputenc}
\usepackage[czech]{babel}
\usepackage[left=2cm,top=3cm,text={17cm,24cm}]{geometry}
\usepackage{hyperref}

\long\def\symbolfootnote[#1]#2{\begingroup%
\def\thefootnote{\fnsymbol{footnote}}\footnote[#1]{#2}\endgroup}

\begin{document}

\begin{center}
    {\LARGE Síťové aplikace a správa sítí}\\[2em]

    {\huge\bf IPv4 a IPv6 konfigurace sítě}\\[2em]
\end{center}

\section*{Cíle cvičení}
\begin{itemize}
    \item Seznámit se s konfigurací IPv4 a IPv6 na OS Linux
\end{itemize}

\section*{Pokyny}
\begin{itemize}
    \item Do zadání nepište, slouží pro další skupiny. Zadání a použité
        konfigurační soubory si lze stáhnout v IS u předmětu ISA.
    \item Po ukončení práce spusťte skript {\tt /root/isa1/clean}
\end{itemize}

\section{Adresy v IPv4 sítí}

\subsection{Formát adresy}
IPv4 adresa je délky 32 bitů a preferovaný zápis má formát {\tt X.X.X.X}, kde
{\tt X} je decimální zápis 8 bitového čísla. Příklad:\\
{\tt
8.8.8.8 \\
127.0.0.1 \\
}

Adresa dvě části:
\begin{itemize}
    \item adresa sítě ({\bf prefix}) a
    \item adresa uzlu.
\end{itemize}
Délká prefixu se zapisuje v desítkovém tvaru za lomítko. {\tt 192.168.0.1/24}

V síti s daným prefixem existují 2 speciální adresy, které nelze použít pro
adresování jednotlivých uzlů:
\begin{itemize}
    \item adresa sítě - adresa uzlu je nulová, např. 192.168.0.0/24
    \item broadcastová adresa - nejvyšší možná adresa uzlu pro daný prefix,
        např. 192.168.0.255/24
\end{itemize}

\section{Adresy v IPv6 sítí}

\subsection{Formát adresy}
IPv6 adresa je délky 128 bitů a zapsaných ve formátu {\tt X:X:X:X:X:X:X:X}, kde
{\tt X} je hexadecimální zápis 16 bitového čísla.
Příklad:\\
{\tt
fedc:ba98:7654:3210:fedc:ba98:7654:3210 \\
1080:0000:0000:0000:0008:0800:200c:417a
}

Preferovaný formát je dále upraven v RFC 5952
\footnote{https://tools.ietf.org/html/rfc5952}, které definuje následující
pravidla:
\begin{itemize}
    \item Nuly na začátku každého 16 bitového čísla je potřeba vynechat (např.
        80 namísto 0080 a 0 namísto 0000).
    \item Více nulových bloků lze nejvýše jednou nahradit znakem {\tt ::}, a
        MUSÍ být nahrazena nejdelší možná posloupnost takových nulových bloků.
        V případe více shodně dlouhých posloupností, nahrazuje se ta nejvíc
        vlevo.
    \item Znak {\tt ::} nesmí nahrazovat samostatný nulový blok.
    \item Znaky "a", "b", "c", "d", "e", "f" hexadecimální soustavy se vždy
        píšou malými písmenami.
\end{itemize}
Adresy z předchozího příkladu tedy budou vypadat nasledovně:\\
{\tt
fedc:ba98:7654:3210:fedc:ba98:7654:3210 \\
1080::8:800:200c:417a
}

Stejně jako v IPv4 má adresa dvě části: \emph{adresa sítě ({\bf prefix})} a
\emph{adresa uzlu}. Délká prefixu se zapisuje v desítkovém tvaru za lomítko
(např. {\tt 1080::/60}).

\subsection{Rozdělení adres}
IPv6 adresy je možno rozdělit podle rozsahu. Typicky může jít o tři
možnosti -- adresy na lince (neprojdou za router), lokální adresy (ULA, nejsou
routovatelné ve veřejné sítí) a veřejné adresy.

\begin{table}[ht!]
    \begin{center}
        \begin{tabular}{l|l|l}
            Význam & Prefix (bitově) & Prefix \\
            \hline\hline
            Veřejné  & {\tt 001} & {\tt 200::/3} \\
            \hline
            Lokální  & {\tt 1111 110} & {\tt FC00::/7} \\
            \hline
            Linkové  & {\tt 1111 1110 10} & {\tt FE80::/10} \\
            \hline
            Multicast & {\tt 1111 1111} & {\tt FF00::/8} \\
            \hline
        \end{tabular}
    \end{center}
\end{table}

\subsection{Lokální IPv6 (ULA) adresy}\label{ula}
Síťová část ULA adresy se stává ze 4 části:
\begin{description}
    \item [Prefix] {\tt fc00::/7}
    \item [L] bit 1 pokud byl prefix přiřazen lokálně, 0 zatím nebyla
        definována, začátek adresy je proto typicky {\tt fd00/8}
    \item [Global ID] identifikátor sítě, měl by být unikátní, standard
        popisuje pseudonáhodný algoritmus pro generování. Unikátnost je
        požadována, aby při spojení více lokálních sítí nebylo třeba žádnou
        přečíslovat.
    \item [Inderface ID] Identifikátor rozhraní, existuje několik variant jak
        jej získat. Původní standard (RFC 3513
        \footnote{https://tools.ietf.org/html/rfc3513}) doporučoval použití
        EUI-64. V současnosti byla tato metoda nahrazena RFC 7217
        \footnote{https://tools.ietf.org/html/rfc7217}. Pro ochranu soukromí
        jsou další specifika automatické generace identifikátorů definovány v
        RFC 4941 \footnote{https://tools.ietf.org/html/rfc4941}.
\end{description}

\begin{table}[ht!]
    \begin{center}
        \begin{tabular}{c|c|c|c|c}
            7 bits & 1 &  40 bits  &  16 bits  & 64 bits \\
            \hline
            1111 110 & L & Global ID & Subnet ID & Interface ID \\
            \hline
        \end{tabular}
    \end{center}
\end{table}

Lokální síť je tedy složená ze tří části:
\begin{itemize}
    \item unikátní prefix délky 48 bitů,
    \item 16 bitový identifikátor podsítě,
    \item adresa uzlu o délce 64 bitů.
\end{itemize}

\section{Postup práce}
Cílem dnešního laboratořního cvičení je vyzkoušet si statickou a dynamickou
konfiguraci IPv4 a IPv6 adres na OS Linux. Do PC se přihlašte jako uživatel
{\bf user} s heslem {\bf user4lab}. Pracovat budete v terminálu jako
uživatel {\bf root}, s heslem {\bf root4lab}. Uživatele můžete v terminálu
změnit pomocí příkazu {\tt su}.

Před samotnou konfigurací je potřeba přepojit počítače kabelem přes patch panel
pro kabely DXX vepředu laboratoře. Na PC budete pracovat s rozhraníma {\bf
eth1}.

Jako první krok je potřeba vypnout službu NetworkManager. Ten se stará o správu
síťových konfiguračních profilů na vyšší úrovni a pokoušel by se měnit naši
konfiguraci.
\begin{verbatim}
systemctl stop NetworkManager.service
\end{verbatim}

\subsection{Statická konfigurace IPv4}
Pro dočasnou statickou konfiguraci IP adres na OS Linux slouží příkaz {\tt ip}.
Pro dnešní cvičení budeme potřebovat následující příkazy:
\begin{itemize}
    \item \verb_ip link set [rozhraní] up/down_ pro zapnutí/vypnutí rozhraní
    \item \verb_ip addr add/del [IP adresa]/[prefix] dev [rozhraní]_ pro
        přidání/odstranení adresy\\z rozhraní
    \item \verb_ip route add default via [IP adresa]_ pro nastavení výchozí
        brány (pro dnešní cvičení není potřeba)
    \item \verb_ip link_, \verb_ip addr_, \verb_ip route_ zobrazí aktuální
        konfiguraci
\end{itemize}

{\bf Zvolte nejdelší možnou masku sítě 192.168.0.0 tak, aby síť obsahovala
prostor pro 100 koncových stanic. Z takto vytvořené sítě si vyberte 2 vhodné
adresy, které nakonfigurujte na koncové uzly Vaší sítě.}

Správnou konfiguraci si ověřte příkazem {\tt ping}.

\subsection{Dynamická konfigurace IPv4 - DHCP}
Pro dynamickou konfiguraci IP adres je potřeba aby byl na síti přístupný
nakonfigurovaný DHCP server a klienti, kteří si o IP adresu požádají. Kromě
přirazování IP adres má DHCP server na starosti i šíření jiných informací
důležitých pro bezproblémovou funkčnost sítě. Jednou takovou informací je IP
adresa doporučeného DNS serveru pro klienty na síti.

Na jednom PC z Vaší sítě, který bude sloužit jako DHCP server, nakonfigurujte
aplikaci ISC DHCP, pomocí konfiguračního souboru \verb_/etc/dhcp/dhcpd.conf_.
Síť si zvolte stejnou jako při statické konfiguraci a IP adresu DNS serveru
zvolte {\bf 10.10.10.1}. Poskytovaný rozsah IP adres zvolte pro 100 stanic,
nezapomeňte, že DHCP server už má IP adresu přirazenou staticky!

Příklad konfigurace DHCP server:
\begin{verbatim}
option domain-name-servers [IP adresy DNS serverů];
subnet [IP adresa sítě] netmask [maska] {
    range [první přiřaditelná IP adresa] [poslední přiřaditelná IP adresa];
}
\end{verbatim}
Dodatečné informace o konfiguračních možnostech najdete v manuálových stránkách
{\tt man dhcpd.conf} případně {\tt man dhcpd}.

Službu DHCP serveru pak třeba spustit pomocí příkazu:
\begin{verbatim}
systemctl start isc-dhcp-server.service
\end{verbatim}

Na druhém PC zrušte statickou konfiguraci IP adres z predcházejícího úkolu a
spusťte DHCP klienta pomocí příkazu {\tt dhclient}. Ten může vypisovat chybu
pro {\tt smbd.service}, kterou ignorujte.
\begin{verbatim}
dhclient -v [rozhraní]
\end{verbatim}

Úspešnou konfiguraci zkontrolujte příkazem {\bf ip addr} a funkčnost ověřte
příkazem {\bf ping}. Informaci o nakonfigurovaném DNS serveru můžete najít v
konfiguračním souboru \verb_/etc/resolv.conf_.

\subsection{Statická konfigurace IPv6}
Statická konfigurace IPv6 se s vyjímkou několika příkazů navíc nikterak neliší.
Zásadní rozdíly jsou v případě konfigurace dynamické. Nezávisle na způsobu
konfigurace je třeba zapnout podporu IPv6, to lze učinit následujícím příkazem:
\begin{verbatim}
sysctl net.ipv6.conf.[rozhraní].disable_ipv6=0
\end{verbatim}

{\bf Zvolte si adresu sítě vhodnou pro použití v privátních lokálních sítích, s
prefixem délky 64 bitů. Prvních 48 bitů zvolte podle popisu v \ref{ula}, pro
vygenerování unikátního Global ID můžete použít web
\url{unique-local-ipv6.com}, zbývajících 16 bitů si můžete zvolit libovolně.  Z
takto vytvořené sítě si vyberte 2 vhodné adresy, které nakonfigurujte na
koncové uzly Vaší sítě. Můžete použít stejné příkazy jako při statické IPv4
konfiguraci.}

Úspešnou konfiguraci zkontrolujte příkazem {\bf ip addr} a funkčnost ověřte
příkazem {\bf ping6}.

\subsection{Dynamická konfigurace IPv6}
Dynamická konfigurace IPv6 se dělí na {\bf bezstavovou} a {\bf stavovou}
konfiguraci. Na dnešním cvičení si vyzkoušíte bezstavovou. Bezstavová
konfigurace byla navržena tak, aby stačilo připojit zařízení do sítě a
automaticky si klient vygeneroval nějakou adresu a ihned mohl komunikovat se
světem. K tomu potřebuje klient znát pouze prefix sítě do které byl připojen. K
tomuto účelu se používají zprávy označované jako Routing Advertisement (RA).
Tyto a ještě další zprávy jsou součásti procesu Neighbor Discovery.

Jeden PC bude opět sloužit jako \uv{router} šířící RA zprávy, k tomuto
použijeme aplikaci {\tt radvd}. Aplikaci je možno nakonfigurovat pomocí souboru
{\tt /etc/radvd.conf}. Tu je zapotřebí nakonfigurovat šíření informací o Vámi
zvoleném prefixu prostřednictvím rozhraní {\bf eth1}.

Příklad konfigurace:
\begin{verbatim}
interface [rozhraní]
{
    AdvSendAdvert on;
    MaxRtrAdvInterval [Max pocet sekund mezi RA zpravami, min 4];
    prefix [prefix]/[delka prefixu]
    {
        AdvOnLink on;
        AdvAutonomous on;
        AdvRouterAddr on;
    };
};
\end{verbatim}
Pokud je nezbytné změnit některé výchozí hodnoty, nebo přidat další, přehled
možností poskytnou manualové stránky {\tt man radvd} a {\tt man radvd.conf}.

Pro správné šíření RA zpráv je potřebné povolit směrování IPv6 provozu:
\begin{verbatim}
sysctl net.ipv6.conf.all.forwarding=1
\end{verbatim}

Nakonec je potřebné službu radvd spustit pomocí příkazu:
\begin{verbatim}
systemctl start radvd.service
\end{verbatim}

Na straně klienta si smažte staticky nakonfigurovanou IP adresu a resetujte
rozhraní pomocí příkazů {\tt ip link set eth1 down} a {\tt ip link set eth1
up}. Pak je potřeba mít pouze zapnutou podporu IPv6 a rozhraní by si malo
samo nakonfigurovat IPv6 adresu pri obdržení nové správy RA.

Součásti balíčku radvd je i apliace {\tt radvdump}, která naslouchá na síťových
rozhraních a tiskne na obrazovku obsah zachycených RA zpráv. {\bf Zkontrolujte
jejich obsah.}

Úspešnou konfiguraci zkontrolujte příkazem {\bf ip addr} a funkčnost ověřte
příkazem {\bf ping6}.

\section{Ukončení práce v laboratoři}
Po povolení vyučujícího můžete spustit skript {\tt /root/isa1/clean}.
%%%%%%%%%%%%%%%%%%%%%%%%%%%%%%%%%%%%%%
\end{document}
%% END OF FILE lab1.tex
