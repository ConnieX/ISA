\section*{Cíle cvičení}
\begin{itemize}
    \item Seznámit se s manuální a dynamickou konfigurací IPv4 a IPv6 na OS Linux
\end{itemize}

\section*{Pokyny}
\begin{itemize}
    \item Do zadání nepište, slouží pro další skupiny. Zadání a použité
        konfigurační soubory si lze stáhnout v IS u předmětu ISA.
    \item Pro práci v laboratoři budeme používat OS Linux, při bootu volba F3.
    \item Uživatelé a hesla pro přihlášení: \texttt{user - user4lab}.
\end{itemize}

\section{Zadání}
Přihlaste se jako uživatel \texttt{user}. Veškeré potřebné příkazy následně
spouštějte jako \texttt{root}.

\subsection{Příprava laboratoře}
Před samotnou konfigurací je potřeba přepojit počítače kabelem přes patch panel
pro kabely DXX (XX je číslo vašeho počítače) vepředu laboratoře. Na PC budete
pracovat s rozhraníma {\bf eth1}.

Jako první krok je potřeba zakázat automatické spuštění služby \texttt{NetworkManager.service} po startu systému. Ta
se stará o správu síťových konfiguračních profilů na vyšší úrovni a pokoušel by
se měnit naši konfiguraci. Informace o správě systémových služeb najdete v
sekci \ref{sluzby-teorie} Teorie.

\subsection{Manuální konfigurace IPv4}

Teorie \emph{IPv4 adresování} je popsaná v sekci \ref{ipv4-teorie} Teorie.
Možnosti manuální konfigurace IP adres jsou v sekci \ref{ip-manual} Teorie.

\begin{enumerate}
    \item Zvolte nejdelší možnou masku sítě \texttt{192.168.0.0} tak, aby síť
        obsahovala prostor pro 100 koncových stanic.
    \item Z takto vytvořené sítě si vyberte 2 vhodné adresy, které
        nakonfigurujte na 2 počítače ve Vaší síti.
    \item Správnou konfiguraci si ověřte příkazem {\tt ping}.
\end{enumerate}

\subsection{Dynamická konfigurace IPv4}

Vaší úlohou bude na jednom počítači nakonfigurovat DHCP server, jehož nastavení se nezmění ani po restartu systému. Na druhém počítači spustíte DHCP klienta, který si od nakonfigurovaného serveru požádá o přidělení vhodné IPv4 adresy. Teorie konfigurace ISC DHCP serveru a použití
DHCP klienta je popsaná v sekci \ref{dhcp} Teorie.

\begin{enumerate}
    \item Počítač, na kterém běží server musí být nakonfigurovaný manuálně tak,
        že na síťovém rozhraní má staticky nakonfigurovanou IP adresu ze správné sítě (proveďte změny v /etc/network/interfaces). Zároveň tato
        adresa nesmí být v rozsahu adres, které poskytuje služba DHCP.
    \item Nakonfigurujte DHCP server tak, aby poskytoval IPv4 adresy pro stejnou
        síť jakou jste si zvolili pro manuální konfiguraci.
    \item Nakonfigurujte DHCP server tak, aby šířil informaci o DNS serveru na
        adrese {\bf 10.10.10.1}.
    \item Povolte automatické spuštění služby ISC DHCP serveru.
    \item Na druhém počítači smažte manuální IP konfiguraci a spusťte DHCP
        klientskou aplikaci.
    \item Správnou konfiguraci si ověřte příkazem {\tt ping} a restartem systému.
    \item Správnou konfiguraci DNS serveru ověřte v konfiguračním souboru
        (\texttt{resolv.conf}).
\end{enumerate}

\subsection{Manuální konfigurace IPv6}
Teorie \emph{IPv6 adresování} je popsaná v sekci \ref{ipv6-teorie} Teorie.
Možnosti manuální konfigurace IP adres jsou stejné jako pro IPv4.

\begin{enumerate}
    \item Zvolte si adresu sítě vhodnou pro použití v privátních lokálních
        sítích, s prefixem délky 64 bitů. Prvních 48 bitů zvolte podle popisu v
        sekci \ref{ula} Teorie, pro vygenerování unikátního Global ID můžete
        použít web \url{https://cd34.com/rfc4193/}, zbývajících 16 bitů (Subnet
        ID) si můžete zvolit libovolně.
    \item Pro zajímavost si můžete unikátnost vygenerovaného Global ID
        zkontrolovat na \url{https://www.sixxs.net/tools/grh/ula/list}.
    \item Z takto vytvořené sítě si vyberte 2 vhodné adresy, které
        nakonfigurujte na 2 počítače ve Vaší síti.
    \item Správnou konfiguraci si ověřte příkazem {\tt ping6}.
\end{enumerate}

\subsection{Dynamická konfigurace IPv6}
Stejně jako u dynamické IPv4 konfigurace bude každý počítač ve dvojici plnit
jinou roli. Jeden počítač bude v roli směrovače vysílat Router Advertisement
zprávy informující o konfiguraci sítě. Druhý počítač bude v roli klientského
zařízení, které si po připojení do sítě na základě RA zprávy nakonfiguruje
správnou IPv6 adresu.

Dynamická IPv6 konfigurace je popsaná v sekci \ref{dyn-ipv6} Teorie.

\begin{enumerate}
    \item Na obou počítačích si ponechte IP adresy nakonfigurované během
        předcházejících úkolů. Pokud došlo ke smazání IPv6 adresy, použijte příkaz \texttt{ip a a fe80::X/64 dev eth1}, kde písmeno X značí číslo PC.
    \item Na jednom z počítačů nakonfigurujte službu \texttt{radvd.service} tak
        aby ve Vaší síti (rozhraní eth1) šířila informace o prefixu sítě, který
        jste si vygenerovali pro manuální konfiguraci.
    \item Na stejném počítači povolte směrování IPv6 provozu pomocí příkazu\\
        \texttt{sysctl net.ipv6.conf.all.forwarding=1}\\
        Tím Váš počítač začne plnit roli směrovače pro IPv6 provoz a bude moci
        šířit RA zprávy.
    \item Povolte automatické spuštění služby \texttt{radvd.service} po startu systému.
    \item Potvrďte správnost nově nakonfigurovaných adres (příkaz \texttt{ip
        address}) a funkčnost sítě pomocí \texttt{ping6}.
    \item Pro zajímavost se můžete podívat na obsah RA zpráv pomocí aplikace
        \texttt{radvdump}, nebo \texttt{wireshark} (filter \texttt{icmpv6.type
        == 134}).
\end{enumerate}

\subsection{Ukončení práce v~laboratoři}
Jakmile máte veškerou práci hotovou, ohlaste se u vyučujícího, který Vám
konfiguraci zkontroluje. Pak můžete spustit skript {\tt /root/isa2/clean} (jako
\texttt{root}).
