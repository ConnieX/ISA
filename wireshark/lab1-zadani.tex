\section*{Cíle cvičení}
\begin{itemize}
	\item Seznámení se se základní prací v~OS Linux.
	\item Seznámení se se základními nástroji pro zjišťování konfigurace zařízení.
	\item Analýza síťového provozu pomocí Wireshark.
\end{itemize}

\section*{Pokyny}
\begin{itemize}
\item Do zadání nepište, slouží pro další skupiny. K zápisu používejte pouze záznamový arch, který vám zůstane.
\item Pro práci v laboratoři budeme používat OS Linux, při bootu volba F3.
\item Uživatelé a hesla pro přihlášení: \texttt{user - user4lab}, \texttt{root - root4lab}.
\end{itemize}

\section{Zadání}
Přihlaste se jako uživatel \texttt{user}. Veškeré potřebné příkazy následně spouštějte jako \texttt{root}.

\subsection{Zjišťování konfigurace}
V případě, že se v OS Linux úplně neorientujete, přečtěte si kapitolu 3 v
laboratorním manuálu --- Základní orientace v Linuxu. V~této části cvičení se
budeme zabývat převážně zjišťováním síťové konfigurace systému. Veškeré potřebné
informace, které budete potřebovat ke splnění této části cvičení, naleznete
v~sekci 3.3 laboratorního manuálu --- Síťová konfigurace. V~případě, že si nejste jistí některým příkazem, neváhejte nahlédnout do manuálové stránky.

\begin{enumerate}
\item Vypište konfiguraci vašeho stroje (IP adresu, masku, síť, broadcastovou adresu).
\item Zobrazte si záznamy v~routovací a ARP tabulce. Zapište pravidlo týkající se defaultní cesty. Následně k~IP ve vypsaném záznamu přiřaďte MAC adresu. Jaký je rozdíl mezi defaultní cestou a defaultní bránou?
\item Otestujte konektivitu k~výchozí bráně a následně konektivitu do internetu.
\item Vypište implicitní servery DNS a soubor, ve kterém jste tuto informaci nalezli.
\item Upravte patřičný soubor tak, aby po spuštění příkazu \texttt{ping google}, byl ping proveden vůči IP adrese 8.8.8.8. Zapište jak a který soubor jste upravili.
\item Vypište aktivní TCP spojení, vyberte jeden záznam, zapište si ho a popište význam jednotlivých položek. Pokud se žádné TCP spojení nezobrazuje, nějaké vygenerujte, například pomocí webového prohlížeče.
\item Zobrazte systémové události pomocí programu \texttt{journalctl}.
\item Zobrazte pouze události týkající se NetworkManager.
\item Pokuste se jako uživatel user spustit Wireshark s pomocí programu \texttt{sudo}. Následně nalezněte v~logu zprávu, která byla zaznamenána v~případě neúspěšného přihlášení.
\end{enumerate}

\subsection{Wireshark}
V~této části cvičení se budeme zabývat analýzou a zachytáváním provozu
v~programu Wireshark. Spuštění Wireshark provedete příkazem \texttt{wireshark}
pod uživatelem \texttt{root}. Veškeré potřebné informace, které budete
potřebovat k~této části cvičení, naleznete v~kapitole 4 laboratorního manuálu
--- Analýza Wireshark.

\begin{enumerate}
  \item Pomocí programu Wireshark začněte \textbf{zachytávat} pouze HTTP komunikaci na výchozím portě
    (výchozí porty je možné nalézt např. v \texttt{/etc/services}). Jakmile začnete zachytávat, spusťte si prohlížeč a načtěte stránku \texttt{http://www.fit.vutbr.cz}. Zachycený provoz uložte do adresáře \texttt{/tmp}. (přípona \texttt{.pcap})
\item Analyzujte zdrojovou a cílovou IP adresu a MAC adresu odeslaného a
  přijatého paketu a doplňte tabulku.
\item Zahajte znovu zachytávání komunikace, nyní bez použití filtru pro HTTP. V příkazové řádce odstraňte ARP záznamy (příkaz \texttt{ip neighbor ...}). Ve wiresharku zobrazte veškerou komunikaci, následně vyfiltrujte pouze ARP a ICMP pakety. Vygenerujte ICMP komunikaci. Analyzujte obsah ARP paketů. Zapište, co jste zadali do filtru.
\item Zachyťte pouze HTTP a DNS provoz (na výchozích portech). Ve webovém prohlížeči zkuste otevřít několik stránek na různých URL adresách. Analyzujte obsah a posloupnost DNS paketů a následných HTTP paketů. Zkuste opakovaně načítat stejnou stránku. Proč v~některých případech není zachycena DNS komunikace?
\item Znovu si otevřete dříve uložený soubor s nešifrovanou komunikaci pomocí
  HTTP, zobrazte si TCP stream této komunikace (\emph{Follow TCP stream}).
    Najděte dotaz, který odpovídá stránce zobrazené v prohlížeči.
 Do odpovědního archu slovně popište význam funkce \emph{Follow TCP stream},
    zamyslete se nad formátem zobrazených dat funkcí \emph{Follow TCP stream} a
    rozdílem oproti výchozímu pohledu na data ve formě paketů.
\end{enumerate}

\subsection{Ukončení práce v~laboratoři}
Jakmile máte veškerou práci hotovou, spusťte jako \texttt{root} skript s~názvem \texttt{clean}, který se nachází v~adresáři \texttt{/root/isa1}.
