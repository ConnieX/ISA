
%%%%%%%%%%%%%%%%%%%%%%%%%%%%%%%%%%%%%%
\section{Zjišťování konfigurace zařízení}
Detailní popis včetně možných voleb a příkladu použití k~jednotlivým níže zmíněným příkazům můžete nalézt v~manuálových stránkách (\texttt{man <příkaz>}).
\subsection{Základní orientace v~Linuxu}
Většina práce v~OS Linux bude probíhat v~terminálu. Terminál na školních PC spustíte pomocí \texttt{Alt+F2}, zde zadáte "\texttt{gnome-terminal}". Případně můžete vybrat aplikaci terminálu z~postraní nabídky.
\subsubsection{Hierarchie souborového systému}
Všechny soubory jsou uloženy v~souborovém systému. Ten je organizován jako invertovaný strom adresářů,
kde kořenovým adresářem je \texttt{root} adresář (označení \texttt{"/"}). Ten obsahuje podadresáře,
kde každý má svůj standardizovaný účel pro zařazení různých souborů.

Některé podadresáře a jejich význam:
\begin{itemize}
				\item \textbf{/etc} - obsahuje konfigurační soubory systému,
				\item \textbf{/var} - proměnlivá boot perzistentní data, dynamicky se mění, obsahuje databáze, cache adresáře, obsah www stránek (\texttt{/var/html/www}), logy (\texttt{/var/log}),
				\item \textbf{/dev} - obsahuje speciální soubory pro přístup k~hardware zařízením.
\end{itemize}

\subsubsection{Základní příkazy}
Některé základní příkazy pro práci v~terminálu OS Linux:
\begin{itemize}
				\item \texttt{cd} - posun v~adresářové hierarchii,
				\item \texttt{ls} - zobrazení obsahu adresáře,
				\item \texttt{cp} - kopírování souborů,
				\item \texttt{mv} - přesun souborů,
				\item \texttt{mkdir} - vytvoření adresáře,
				\item \texttt{touch} - vytvoření prázdného souboru,
				\item \texttt{head} - zobrazení x řádků od začátku souboru,
				\item \texttt{tail} - zobrazení x řádků z~konce souboru,
				\item \texttt{cat} - zobrazení obsahu souboru,
				\item \texttt{grep} - filtrování/hledání v~textu,
				\item \texttt{sed} - editace textu v~příkazové řádce,
				\item \texttt{awk} - skenování a zpracování textu,
				\item \texttt{ps} - zobrazení informací o~běžících procesech.
\end{itemize}

Mezi command line editory patří například \texttt{nano} nebo \texttt{vim}. Kromě nich můžete taktéž použít grafické editory, například \texttt{gedit}.

\subsubsection{Uživatelé}
Každý uživatel v~OS Linux má své jedinečné \texttt{uid}. Každý proces (program)
v~systému je spuštěn pod nějakým uživatelem. Každý soubor je vlastněn uživatelem
a omezen přístupovými právy. Tato přístupová práva se vztahují taktéž na procesy
spuštěné daným uživatelem.

\begin{itemize}
				\item \texttt{root} - super uživatel, má veškerá práva a kontrolu nad systémem.
				\item \texttt{user} - pouze základní správa, bez možnosti zasahovat do systémového nastavení.
				\item \texttt{sudo} - delegace některých práv super uživatele na normálního uživatele (nastavení v~\texttt{/etc/sudoers}).
\end{itemize}

Při správě OS se pokud možno snažíme vyhnout používání systému jako \texttt{root}. K~tomuto slouží \texttt{sudo},
kde má daný uživatel přesně specifikováno, které operace smí se systémem provádět, a je pak lépe dohledatelné,
kdo co nastavil.

Přepínání uživatele:
\texttt{su [user]}
\begin{itemize}
				\item \texttt{su} - přepnutí na uživatele \texttt{root}.
				\item \texttt{su user} - přepnutí na uživatele \texttt{user}.
\end{itemize}

Spouštění příkazů jako \texttt{sudo}:
\texttt{sudo command...}


\subsection{Síťová konfigurace zařízení}
Ke zjišťování síťové konfigurace na daném zařízení můžeme použít několik nástrojů,
které jsou na OS Linux k~dispozici.

\subsubsection{Zobrazení konfigurace}
Jedním ze základních příkazů je \texttt{ip}. Pomocí tohoto příkazu můžeme zjistit
konfiguraci všech síťových zařízení (fyzických i virtuálních), obsah routovací tabulky aj. Manuálové stránky pro dané možnosti zobrazíte jako \texttt{ip-<volba>}.

\begin{itemize}
				\item \texttt{ip address} - zobrazí adresu na všech síťových zařízení.
				\item \texttt{ip route} - zobrazí všechna pravidla v~routovací tabulce.
				\item \texttt{ip link} - vylistuje všechna síťová zařízení.
				\item \texttt{ip neighbor} - zobrazí obsah ARP záznamů.
\end{itemize}

Tento nástroj mimo jiné slouží také ke konfiguraci, ne jen k~jejímu zobrazení. \footnote{Obdobou tohoto nástroje je \texttt{ifconfig}.}

%- nmcli

Alternativou k~zobrazení obsahu routovací a ARP tabulky jsou tyto příkazy:
\begin{itemize}
\item \texttt{netstat} - zobrazí mimo jiné obsah routovací tabulky (volba \texttt{-rn})
\item \texttt{arp} - zobrazí obsah ARP tabulky (také umožňuje provádět správu, například odstranit záznamy)
\end{itemize}


K~zobrazení aktuálně běžících spojení včetně protokolu, portu, zdrojové a cílové adresy slouží příkaz \texttt{sockstat} (\texttt{ss}).

\subsubsection{Konfigurační soubory a logy}
Na OS Linux je veškeré nastavení systému a služeb uloženo v~adresáři \texttt{/etc}. Jednotlivé služby zde mají svůj konfigurační soubor s~příslušným názvem. Pro lepší přehlednost je možné vytvářet více konfiguračních souborů, pro tyto účely zde pak existují adresáře \texttt{"nazevsluzby.d/"} (například \texttt{/etc/rsyslog.d/}). Veškeré nastavení v~souborech v~těchto adresářích je pak aplikováno na danou službu.

Některé ze základních podstatných konfiguračních souborů:
\begin{itemize}
				\item \texttt{/etc/hostname} - obsahuje hostname systému.
				\item \texttt{/etc/hosts} - obsahuje mapování hostname na IP adresu (zde můžete libovolné adrese přiřadit hostname, následné použití tohoto hostname předchází samotnému vyhledání pomocí DNS resolveru).
				\item \texttt{/etc/host.conf} - obsahuje konfiguraci specifickou pro resolver.
				\item \texttt{/etc/resolv.conf} - obsahuje direktivy specifikující výchozí servery, na které je poslán dotaz pro překlad doménového jména na IP adresu.
				\item \texttt{/etc/rsyslog.conf} - obsahuje nastavení syslogu, rozřazování jednotlivých zpráv do příslušných logovacích souborů.
\end{itemize}


Veškeré podstatné události jsou zaznamenávány do logů. Všechny logovací soubory jsou uloženy v~adresáři \texttt{/var/log}. Tyto soubory (logy) jsou cyklicky promazávány po určitém čase či po dosažení určité velikosti. Dobu, po kterou jsou logy udržovány, lze nastavit v~konfiguračním souboru\\\texttt{/etc/logrotate.conf}.

Některé významné logovací soubory:
\begin{itemize}
				\item \texttt{/var/log/messages} - obsahuje obecné zprávy systému (bez kritických a debugovacích).
				\item \texttt{/var/log/auth.log} - zde jsou uloženy zprávy týkající se autentizace uživatelů.
				\item \texttt{/var/log/kern.log} - ukládá zprávy týkající se jádra OS.
\end{itemize}

Tyto logovací soubory můžeme prohlížet přímo pomocí standardních zobrazovacích příkazů.

Systémové události můžeme také prohlížet pomocí nástroje \texttt{journalctl}, ten nám umožní procházet podrobnější informace daných zpráv, filtrovat a vyhledávat.

Příklad použití \texttt{journalctl}:
\begin{itemize}
				\item \texttt{journalctl -n <x>} - zobrazí posledních x záznamů.
				\item \texttt{journalctl -p <priority>} - zobrazí pouze záznamy s~danou syslog prioritou.
				\item \texttt{journalctl -u <unit|patern>} - umožní vyfiltrovat pouze záznamy s~daným paternem/službou.
				\item \texttt{journalctl -f} - zobrazí kontinuálně posledních 10 událostí.
				\item \texttt{journalctl --since <whenstart> --until <whenend>} - zobrazí záznamy v~daném rozmezí.
\end{itemize}


\subsubsection{Aktivní zjišťování}
Zjišťování dostupnosti zařízení a konektivity. K~tomuto účelu poslouží příkazy:

\begin{itemize}
				\item \texttt{ping ipv4|domain} - zašle ICMP ECHO\_REQUEST k~danému hostu (pro ipv4).
				\item \texttt{ping6 ipv6|domain} - zašle ICMP ECHO\_REQUEST k~danému hostu (pro ipv6).
				\item \texttt{traceroute ipv4|domain} - zobrazí cestu, kudy packet prochází skrz síť k~danému hostu (pro ipv4).
				\item \texttt{traceroute6 ipv6|domain} - zobrazí cestu, kudy packet prochází skrz síť k~danému hostu (pro ipv6).
				\item \texttt{tcptraceroute ipv4/ipv6|domain} - \texttt{traceroute} používající TCP.
\end{itemize}

Mimo základní výše uvedené příkazy \texttt{ping} a \texttt{traceroute}
existuje užitečný nástroj \texttt{nmap}.

\begin{itemize}
				\item \texttt{nmap} - umožňuje pomocí různých protokolů aktivně zjišťovat otevřené porty, aktivní hosty na dané síti apod.
\end{itemize}


Pro přihlášení na vzdálený host můžeme použít dva nástroje:
\begin{itemize}
				\item \texttt{telnet ip} - nešifrované přípojení
				\item \texttt{ssh ip} - šifrované přípojení
\end{itemize}



\section{Analýza Wireshark}
Wireshark je aplikace sloužící k~analýze protokolů a zachytávání paketů. Zachytává síťový provoz z~jednotlivých síťových zařízení (jak fyzických, tak virtuálních) a umožňuje inspekci jednotlivých polí v~daných paketech.

Alternativou ke grafickému Wireshark je command line nástroj \texttt{tcpdump}. Wireshark má oproti \texttt{tcpdump} například integrované volby pro řazení, filtrování a jiné.


\subsubsection{Zachytávání provozu}
Zachytávat provoz můžeme z~libovolného síťového zařízení, dokonce z~více zařízení najednou.

Odchytíme pomocí \texttt{Capture -> Interfaces...} a zde vybereme zařízení, které chceme snímat. Spustíme pomocí \texttt{Start}. Takto odchytíme veškerý provoz.

Existuje zde volba filtrace provozu již při zachytávání. Tato vlastnost může být užitečná při dlouhodobějším zachytávání, především z~hlediska redukce množství ukládaných paketů.

Paketové filtry můžeme aplikovat před spuštěním zachytávání. Použijeme \texttt{Options} a v~poli \texttt{Capture Filter:} můžeme zvolit jeden z~předdefinovaných filtrů, případně specifikovat vlastní.

Odchycený provoz můžeme uložit do souboru (přípona \texttt{.pcap}). Následně je možné tyto soubory znovu analyzovat.


\subsubsection{Filtrovací operátory}
V některých situacích, napříkad pokud zachytáváme veškerý provoz, je velmi vhodné použít filtrování. Jelikož zde pracujeme s velkým množstvím různých paketů, můžeme si tak vyfiltrovat pouze ty relevantní. K tomu slouží tzv. display filtry, jejichž syntax je popsána v Tabulce 1.
\begin{center}
\begin{table}[h]
\centering
\def\arraystretch{1.2}
\begin{tabular}{|l|l|}
\hline
\textbf{Porovnávání} & \texttt{==, >=, <=, !=, contains}\\
\textbf{Logické operátory} & \texttt{||, or, \&\&, and, !, not}\\
\textbf{Kombinace filtrů} & \texttt{(ip.src==192.168.0.105 and udp.port==53)}\\
\textbf{Filtrování na základě existence pole} & \texttt{http.cookie or http.set\_cookie}\\
\textbf{Filtrování specifických bytů} & \texttt{eth.src[4:2]==22:1b}\\
\textbf{Regex filtrování} & \texttt{http.host \&\& !http.host matches "\.com\$"}\\
\hline
\end{tabular}
\caption{Filtrovací operátory}
\end{table}
\end{center}


\subsection{Flow graph}
Wireshark umožňuje zobrazit vybranou komunikaci v~\texttt{flow graph}. Zde můžeme přehledně vidět, které stroje, rozlišené IP adresou, spolu komunikují a kdo komu, zasílá kterou zprávu v~jakém pořadí.
Flow graph si zobrazíme pomocí \texttt{Statistics -> Flow Graph..}, zde si vybereme, co chceme zobrazit.
\subsection{Zobrazení streamu}
Zachycenou komunikaci vidíme v~podobě jednotlivých paketů. Pokud například zachytíme HTTP komunikaci, jeden paket nese většinou pouze část obsahu HTTP zprávy. V tomto případě můžeme pro zobrazení celé zprávy (celého streamu) použít volbu \texttt{Flow <protokol> Stream}. Zobrazení provedeme tak, že vybereme jeden zachycený paket, označíme ho a následně \texttt{Analyze -> Follow <protokol> Stream}.
\subsection{Čas}
V rámci zachytávání paketů nese každý záznam informaci o čase. Defaultně je zde čas zobrazen v~sekundách, od počátku zachytávání komunikace. Nicméně v~\texttt{View -> Time Display Format} můžeme vybrat konkrétní zobrazení času a zjistit tak například, ve který den byl daný paket zachycen.
