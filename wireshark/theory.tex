
%%%%%%%%%%%%%%%%%%%%%%%%%%%%%%%%%%%%%%
\section{Zjišťování konfigurace zařízení}
Detailní popis, včetně možných voleb a příkladu použití, k jednotlivým níže zmíněným příkazům můžete nalézt v manuálnových stránkách (\texttt{man <příkaz>}).
\subsection{Základní orientace v Linuxu}
\subsubsection{Hierarchie souborového systému}
Všechny soubory jsou uloženy v souborovém systému. Ten je organizován jako invertovaný strom adresářů,
kde kořenovým adresářem je \texttt{root} adresář (označení \texttt{"/"}). Tento obsahuje podadresáře,
kde každý má svůj standardizovaný účel pro zařazení různých souborů.

Některé podadresáře a jejich význam:
\begin{itemize}
				\item \textbf{/etc} - obsahuje konfigurační soubory systému
				\item \textbf{/var} - proměnlivá boot persistentní data, dynamicky se mění, obsahuje databáze, cahe adresáře, obsah www stránek (\texttt{/var/html/www}), logy (\texttt{/var/log})
				\item \textbf{/dev} - obsahuje speciální soubory pro přístup k hardware zařízením
\end{itemize}
\subsubsection{Základní příkazy}
Některé základní příkazy pro práci v terminálu OS Linux:
\begin{itemize}
				\item \texttt{cd} - posun v adresářové hierarchii
				\item \texttt{ls} - zobrazení obsahu adresáře
				\item \texttt{cp} - kopírování souborů
				\item \texttt{mv} - přesun souborů
				\item \texttt{mkdir} - vytvoření adresáře
				\item \texttt{touch} - vytvoření prázdného souboru
				\item \texttt{head} - zobrazí x řádků od začátku souboru
				\item \texttt{tail} - zobrazí x řádku z konce souboru
				\item \texttt{cat} - zobrazí obsah souboru
				\item \texttt{grep} - filtrování/hledání v textu
				\item \texttt{sed} - editace textu v příkazové řádce
				\item \texttt{awk} - skenování a zpracování textu
				\item \texttt{ps} - zobrazí informace o běžících procesech
\end{itemize}

Command line editory:
\begin{itemize}
				\item \texttt{vim}
				\item \texttt{nano}
\end{itemize}

\subsubsection{Uživatelé}
Každý uživatel v OS Linux má své jedinečné \texttt{uid}. Každý proces (program)
v systému je spuštěn pod nějakým uživatelem. Každý soubor, je vlastněn uživatelem
a omezen přístupovými právy. Tyto přístupová práva se vztahují taktéž na procesy
spuštěné daným uživatelem.

\begin{itemize}
				\item \texttt{root} - super uživatel, má veškerá práva a kontrolu nad systémem
				\item \texttt{user} - pouze základní správa, bez možnosti zasahovat do systémového nastavení
				\item \texttt{sudo} - delegace některých práv super uživatele na normálního uživatele (nastavení v \texttt{/etc/sudoers})
\end{itemize}

Při správě OS se pokud možno snažíme vyhnout nastavování systému jako \texttt{root}. K tomuto slouží \texttt{sudo}
kde daný uživatel má přesňě specifikováno, které operace smí se systémem prováďět, a je pak lépe dohledatelné,
kdo a co nastavil.

Přepínání uživatele:
\texttt{su [user]}
\begin{itemize}
				\item \texttt{su} - přepnutí na uživatele \texttt{root}
				\item \texttt{su user} - přepnutí na uživatele user
\end{itemize}

Spouštění příkazů jako \texttt{sudo}:
\texttt{sudo command...}


\subsection{Síťová konfigurace zařízení}
Ke zjišťování síťové konfigurace na daném zařízení můžeme použít několik toolů,
které jsou na OS Linux k dispozici.

\subsubsection{Zobrazení konfigurace}
Jedním ze základních příkazů je \texttt{ip}, pomocí tohoto příkazu můžeme zjistit
konfiguraci všech síťových zařízení (fyzických i virtuálních), obsah routovací tabulky aj.

\begin{itemize}
				\item \texttt{ip address} - zobrazí adresu na všech síťových zařízeních
				\item \texttt{ip route} - zobrazí všechna pravidla v routovací tabulce
				\item \texttt{ip link} - vylistuje všechna síťová zařízení
\end{itemize}

Tento tool mimo jiné slouží také ke konfiguraci, ne jen k zobrazení konfigurace. \footnote{Obdobou tohoto toolu je \texttt{ifconfig}.}

%- nmcli

Pro zobrazení obsahu routovací a arp tabulky, můžeme použít příkazy:
\begin{itemize}
\item \texttt{netstat -rn} - vypíše obsah routovací tabulky
\item \texttt{arp -a} - zobrazí obsahu arp tabulky
\end{itemize}


K Zobrazení aktuálně běžících spojení včetně protokolu, portu, zdrojové a cílové adresy slouží příkaz \texttt{sockstat} (\texttt{ss}).

\subsubsection{Konfigurační soubory a logy}
Na OS Linux je veškeré nastavení systému a služeb uloženo v adresáři \texttt{/etc}. Jednotlivé služby zde mají svůj konfigurační soubor s příslušným názvem. Pro lepší přehlednost je možné vytvářet více konfiguračních souborů, pro tyto účely zde pak existují adresáře \texttt{"nazevsluzby.d/"} (například \texttt{/etc/rsyslog.d/}). Veškeré nastavení v souborech v těchto adresářích je pak aplikováno na danou službu.

Některé ze základních podstatných konfiguračních souborů:
\begin{itemize}
				\item \texttt{/etc/hostname} - obsahuje hostname systému
				\item \texttt{/etc/hosts} - obsahuje mapování hostname na ip adresu (zde můžete libovolné adrese přiřadit hostname, následné použití tohoto hostname, předchází samotnému vyhledání pomocí DNS resolveru)
				\item \texttt{/etc/host.conf} - obsahuje konfiguraci specifickou pro resolver
				\item \texttt{/etc/resolv.conf} - obsahuje direktivy, které specifikují výchozí servery, kterých se má být dotazováno pro překlad doménového jména na ip adresu
				\item \texttt{/etc/rsyslog.conf} - obsahuje nastavení syslogu, rozřazování jednotlivých zpráv do příslušných logovacích souborů
\end{itemize}


Veškeré podstatné události, jsou zaznamenávány do logů. Všechny logovací soubory jsou uloženy v adresáři \texttt{/var/log}. Tyto soubory (logy) jsou cyklicky promazávány po určitém čase či po dosažení určité velikosti. Doba po kterou jsou logy udržovány lze nastavit v konfiguračním souboru \texttt{/etc/logrotate.conf}.

Některé významné logovací soubory:
\begin{itemize}
				\item \texttt{/var/log/messages} - obsahuje obecné zprávy systému (bez kritických a debugovacích)
				\item \texttt{/var/log/auth.log} - zde jsou uloženy zprávy týkající se autentizace uživatelů
				\item \texttt{/var/log/boot.log} - ukládá zprávy ohledně bootování systému, vždy poslední nabootování
				\item \texttt{/var/log/mail.log} - zprávy týkající se emailové komunikace
\end{itemize}

Tyto logovací soubory můžeme prohlížet přímo pomocí standardních zobrazovacích příkazů.

Systémové eventy můžeme také prohlížet pomocí nástroje \texttt{journalctl}, ten nám umožní procházet podrobnější informace daným zprávám, filtrovat a vyhledávat.

Příklad použití \texttt{journalctl}:
\begin{itemize}
				\item \texttt{journalctl -n 5} - zobrazí posledních 5 záznamů
				\item \texttt{journalctl -p err} - zobrazí pouze záznamy s prioritou error
				\item \texttt{journalctl -f} - zobrazuje kontinuálně posledních 10 událostí
				\item \texttt{journalctl --since <whenstart> --until <whenend>} - zobrazí záznamy v daném rozmezí
\end{itemize}


\subsubsection{Aktivní zjišťování}
Zjišťování dostupnosti zařízení a konektivity. K tomuto účelu poslouží příkazy.

\begin{itemize}
				\item \texttt{ping ipv4|domain} - zašle ICMP ECHO\_REQUEST k danému hostu (pro ipv4)
				\item \texttt{ping6 ipv6|domain} - zašle ICMP ECHO\_REQUEST k danému hostu (pro ipv6)
				\item \texttt{traceroute ipv4|domain} - zobrazí cestu kudy packet prochází skrz síť k danému hostu (pro ipv4)
				\item \texttt{traceroute6 ipv6|domain} - zobrazí cestu kudy packet prochází skrz síť k danému hostu (pro ipv6)
				\item \texttt{tcptraceroute ipv4/ipv6|domain} - \texttt{traceroute} používající tcp
\end{itemize}

Mimo základní výše zmíněné příkazy \texttt{ping} a \texttt{traceroute},
existuje užitečný tool \texttt{nmap}.

\begin{itemize}
				\item \texttt{nmap} - umožňuje aktivně zjišťovat otevřené porty, aktivní hosty na dané síti apod., pomocí různých protokolů
\end{itemize}


Pro přihlášení na vzdálený host můžeme použít dva tooly:
\begin{itemize}
				\item \texttt{telnet ip} - nešifrovaně
				\item \texttt{ssh ip} - šifrovaně
\end{itemize}



\section{Analýza Wireshark}
Wireshark je aplikace sloužící k analýze protokolů a zachytávání paketů. Zachytává síťový provoz z jednotlivých síťových zařízení (jak fyzických tak virtuálních) a umožňuje inspekci jednotlivých polí v daných paketech.

Alternativou ke grafickému Wireshark je commandline tool \texttt{tcpdump}. Plní stejnou funkci, Wireshark má oproti \texttt{tcpdump} například integrované volby pro řazení a filtrování.


\subsubsection{Zachytávání provozu}
Zachytávat provoz můžeme z libovolného síťového zařízení, dokonce z více zařízení najednou.

Odchytíme takto \texttt{Capture -> Interfaces...}, zde vybereme zařízení, které chceme snímat. A spustíme pomocí \texttt{Start}. Takto odchytíme veškerý provoz.

Existuje zde volba filtrace provozu již při zachytávání. Tato vlastnost může být užitečné při dlouhodobějším zachytávání, především z hlediska redukce množství ukládaných paketů.

Paketové filtry můžeme aplikovat před spuštěním zachytávání. Je zde volba \texttt{Options}, v poli \texttt{Capture Filter:} můžeme zvolit jeden z předdefinovaných filtrů, případně specifikovat vlastní.

Formát filtru je BPF
%======================================

Odchycený provoz můžeme uložit do souboru (přípona \texttt{.pcap}). Následně je možné tyto soubory znovu analyzovat.


\subsubsection{Filtrovací operátory}
\begin{itemize}
\item \textbf{Porovnávání:} \texttt{==, >=, <=, !=, contains}
\item \textbf{Logické operátory:} \texttt{||, or, \&\&, and, !, not}
\item \textbf{Kombinace filtrů:} \texttt{(ip.src==192.168.0.105 and udp.port==53) or tcp.port==80}
\item \textbf{Filtrování na základě existence pole:} \texttt{http.cookie or http.set\_cookie}
\item \textbf{Filtrování specifických bytů:} \texttt{eth.src[4:2]==22:1b}
\item \textbf{Regex filtrování:} \texttt{http.host \&\& !http.host matches "\.com\$"}
\end{itemize}

\subsection{Flow graph}
