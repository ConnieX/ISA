\section{Konfigurace síťování koncových zařízení}
\label{ipconfig}

\subsection{Manuální konfigurace IP Adres}\label{ip_manual}
Pro dočasnou konfiguraci IP adres na OS Linux slouží příkaz {\tt ip}. Pro
dnešní cvičení budeme potřebovat následující příkazy:
\begin{itemize}
    \item \verb_ip link set [rozhraní] up/down_ pro zapnutí/vypnutí rozhraní
    \item \verb_ip addr add/del [IP adresa]/[prefix] dev [rozhraní]_ pro
        přidání/odstranění adresy z~rozhraní
    \item \verb_ip addr flush dev [rozhraní]_ pro odstranění všech adres
z~rozhraní
    \item \verb_ip route add default via [IP adresa]_ pro nastavení výchozí
        brány
    \item \verb_ip link_, \verb_ip addr_, \verb_ip route_ zobrazí aktuální
        konfiguraci
\end{itemize}

\subsection{Dynamická konfigurace IPv4 - DHCP}\label{dhcp}
Pro dynamickou konfiguraci IP adres je potřeba aby byl na síti přístupný
nakonfigurovaný DHCP server a klienti, kteří si o~IP adresu požádají. Kromě
přiřazení IP adres má DHCP server na starosti i šíření jiných informací
důležitých pro bezproblémovou funkčnost sítě. Jednou takovou informací je IP
adresa doporučeného DNS serveru pro klienty na síti.

Na Vašich počítačích máte nainstalovanou serverovou aplikaci ISC DHCP, která
poskytuje služby DHCP serveru. Aplikace se konfiguruje pomocí souboru
\verb_/etc/dhcp/dhcpd.conf_. Systémová služba se jmenuje
\texttt{isc-dhcp-server.service}.

Příklad obsahu konfiguračního souboru:
\begin{verbatim}
option domain-name-servers [IP adresy DNS serverů];
subnet [IP adresa sítě] netmask [maska] {
    range [první přiřaditelná IP adresa] [poslední přiřaditelná IP adresa];
}
\end{verbatim}
Dodatečné informace o~konfiguračních možnostech najdete v~manuálových stránkách
{\tt man dhcpd.conf} případně {\tt man dhcpd}.

DHCP klient je implementován v~aplikaci \texttt{dhclient}. Může být spuštěn bez
parametrů pro všechna aktivní síťová rozhraní nebo lépe s~jménem konkrétního
síťového rozhraní, které má být konfigurováno:
\begin{verbatim}
dhclient -v [rozhraní]
\end{verbatim}
Argument \texttt{-v} zajistí, že aplikace vypíše detailní informace.
Chybu týkající se {\tt smbd.service} někdy zobrazovanou v~laboratoři můžete ignorovat.

\subsection{Dynamická konfigurace IPv6}
\label{dynipv6}

Dynamická konfigurace IPv6 se dělí na {\bf bezstavovou} a {\bf stavovou}
konfiguraci. Na cvičeních se budeme zabývat jen bezstavovou. Bezstavová
konfigurace byla navržena tak, aby stačilo připojit zařízení do sítě a
automaticky si klient vygeneroval nějakou adresu a ihned mohl komunikovat se
světem. K~tomu klient potřebuje znát prefix sítě do které byl připojen.
K~tomuto účelu se používají zprávy označované jako Routing Advertisement (RA).
Tyto a ještě další zprávy jsou součásti procesu Neighbor Discovery~\cite{rfc4861}.

Na počítačích v~laboratoři je nainstalovaná systémová služba {\tt radvd.service},
která je schopna vysílat zprávy RA. Aplikaci je možno nakonfigurovat pomocí
souboru {\tt /etc/radvd.conf}.

Příklad konfigurace:
\begin{verbatim}
interface [rozhraní]
{
    AdvSendAdvert on;
    MaxRtrAdvInterval [Max pocet sekund mezi zpravami RA, min 4];
    prefix [prefix]/[delka prefixu]
    {
        AdvOnLink on;
        AdvAutonomous on;
        AdvRouterAddr on;
    };
};
\end{verbatim}
Přehled všech možností konfigurace poskytnou manualové stránky {\tt man radvd}
a {\tt man radvd.conf}.

Pro správné šíření RA zpráv je potřebné povolit směrování IPv6 provozu:
\begin{verbatim}
sysctl net.ipv6.conf.all.forwarding=1
\end{verbatim}

Součásti balíčku radvd je i aplikace {\tt radvdump}, kterou možno použít pro
analýzu -- naslouchá na síťových rozhraních a tiskne na obrazovku obsah
zachycených RA zpráv.
