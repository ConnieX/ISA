%%%%%%%%%%%%%%%%%%%%
\section{Secure Shell}
\label{ssh}

Základní funkcí protokolu Secure Shell (SSH) \cite{rfc4253} je umožnění bezpečného přístupu
 ke vzdálenému počítači přes nezabezpečenou síť. Díky tomu, že je protokol SSH navržen obecně,
 lze pomocí něj zabezpečovat i další služby, jako je např. X Window, přístup ke vzdálenému
 souborovému systému (SFTP, sshfs, scp), tunelování portů TCP apod. Protokol SSH zajišťuje
 šifrování dat, autentizaci, integritu dat a volitelně také kompresi přenášených dat.

V~rámci předmětu ISA se zaměříme pouze na malou část možností, které protokol SSH přináší. V~rámci
 cvičení si vyzkoušíme protokol SSH pro terminálový přístup ke vzdálenému stroji a ukážeme si
 využití přihlašování ke vzdálenému počítači pomocí klíčů. Dále si vyzkoušíme, jak využít SSH
 k~vytvoření jednoduché HTTP proxy.

Jednou z~nejčastěji používaných aplikací pro využití protokolu SSH je sada programů OpenSSH.
 Balíček programů obsahuje kromě klienta {\tt ssh} i serverovou aplikaci {\tt sshd}, program
 pro generování SSH klíčů {\tt ssh-keygen}, agenta pro usnadnění práce s~SSH klíči {\tt ssh-agent}
 a další. My budeme předpokládat, že na počítačích, na které se budeme snažit připojit
 již běží SSH server {\tt sshd}. Konfigurace tohoto programu je nad rámec tohoto manuálu. Zájemci
 mohou nalézt podrobnější informace v~manuálové stránce {\tt sshd\_config(5)}, či v~jiných návodech.

\subsection{Připojení ke~vzdálenému počítači}

Pro připojení se k~počítači pojmenovaném \emph{h01} a otevření příkazového řádku na vzdáleném
 stroji je možné použít příkaz:

\begin{verbatim}
ssh h01
\end{verbatim}

Příkaz ssh má celou řadu parametrů, které jsou detailně popsány v~manuálové stránce {\tt ssh(1)}.
 Z~těch nejčastěji používaných zmíníme alespoň změnu uživatelského jména ({\tt -l}), specifikování
 TCP portu vzdáleného serveru ({\tt -p}), zvýšení výřečnosti programu ({\tt -v}, tento parametr
 lze použít i vícekrát), zapnutí tunelování protokolu X Windows ({\tt -X}, {\tt -Y})
 a přesměrování portů ({\tt -L}). Často zadávané parametry se specifikují v konfiguračním souboru
 \verb|~/.ssh/config|. Popis tohoto souboru obsahuje manuálová stránka {\tt ssh\_config(5)}.

Po připojení ke vzdálenému serveru jsou informace o~použitém spojení dostupné např. v~rámci
 proměnných prostředí. Proměnné prostředí související s~protokolem SSH je možné zobrazit
 příkazem {\tt env | grep SSH}. Následující výpis ukazuje příklad proměnných po připojí
 k~serveru {\tt merlin} protokolem IPv6:

\begin{verbatim}
local $ ssh merlin6.fit.vutbr.cz 
merlin $ env | grep SSH
SSH_CLIENT=2001:67c:1220:80c:e138:4d11:c04c:c675 54514 22
SSH_TTY=/dev/pts/30
SSH_CONNECTION=2001:67c:1220:80c:e138:4d11:c04c:c675 \
  54514 2001:67c:1220:8b0::93e5:b013 22
\end{verbatim}

Z~výpisu vidíme, že připojení bylo realizováno na IPv6 adresu {\tt 2001:67c:1220:8b0::93e5:b013}
 z~počítače s~adresou {\tt 2001:67c:1220:80c:e138:4d11:c04c:c675}. Byl použit zdrojový port č. 54514,
 na straně serveru byl použit standardní protokol 22. Po připojení využíval vzdálený uživatel
 terminál č. 30. Odhlášení ze vzdáleného počítače probíhá standardními prostředky pro ukončení
 shellu, např. příkaz {\tt exit}, nebo vložení konce souboru klávesovou zkratkou {\tt Ctrl-D}.

\subsection{Kopírování souborů mezi počítači}

Pro kopírování souborů protokolem SSH je často využívaná utilita {\tt scp}. Cesta na vzdáleném
 serveru je specifikována v~následujícím formátu: {\tt uživatel@jménoserveru:cesta}.
 Následující příkaz zkopíruje lokální soubor isa na vzdálený počítač h01 do
 složky fit umístěné v~domovské složce uživatele student.

\begin{verbatim}
scp isa student@h01:fit/
\end{verbatim}

\subsection{SSH jako proxy a tunelování portů}
Díky velmi obecné implementaci lze protokolem SSH tunelovat jakýkoli typ provozu. Je tedy možné
 např. zajistit šifrování provozu po určitou část komunikace, konkretně po stanici, ke které
 se připojujeme SSH protokolem, nebo přesměrovat provoz z~některého portu na jiný port. Následujcí
 příkaz nám otevře SSH spojení, které pak můžeme použít jako proxy ve webovém prohlížeči.

\begin{verbatim}
  ssh -D 12345 -N student@sshserver
\end{verbatim}

Po autentizaci stačí pak v~prohlížeči zadat jako nastavení proxy server localhost s~portem 12345
 a~provoz bude směrován nejprve na localhost, poté půjde ssh spojením na sshserver, kde bude převeden
 na běžný HTTP/HTTPS provoz a~poslán dále. 

\subsection{Klíče protokolu SSH}

Klíče protokolu SSH mají několik výhod. Díky nim není nutné posílat heslo pro přístup ke vzdálenému
 stroji přes síť, byť v~šifrované podobě. Délka používaných klíčů znesnadňuje potenciálnímu
 útočníkovi útok hrubou silou, protože síla klíče bývá typicky vyšší než síla běžně používaných
 hesel. Ve spojení s~agentem pro správu klíče může uživatel přistupovat ke vzdálenému serveru bez
 nutnosti opakovaného zadávání hesla. Agent může být nastaven, aby pro použitý klíč nevyžadoval
 heslo po zbytek sezení, či po určitý počet minut.

SSH klíč se obvykle generují utilitou {\tt ssh-keygen}. Po spuštění bez parametrů je vytvořen pár
 klíčů algoritmem RSA o~délce 2048\,B. Uživateli je nabídnuto umístění souboru,
 které může změnit. Dále je uživatel požádán o~zadání passfráze, od které se očekává, že bude
 silnější než heslo. Parametr {\tt -t} nastavuje jiný typ šifrovacího algoritmu,
 parametr {\tt -b} mění délku klíče, parametrem {\tt -f} specifikuje umístění souboru, parametr
 {\tt -C} upravuje popis klíče, další parametry popisuje manuálová stránka {\tt ssh-keygen(1)}.
 Následující příklad vygeneruje klíč o~délce 4096\,B, který nebude chráněn passfrázi:

\begin{verbatim}
ssh-keygen -t rsa -b 4096 -N "" -f ~/.ssh/nopass -C nopass
\end{verbatim}

Po vytvoření klíčů vzniknou dva soubory. Ten který je bez přípony {\tt .pub} je soukromý klíč,
 který by měl zůstat tajný a uživatel, který jej vytvořil by jej neměl dále distribuovat.
 Soubor s~příponou {\tt .pub} je určen pro další distribuci, protože data zašifrovaná tímto
 klíčem dešifruje pouze tajný soukromý klíč.

\subsection{Konfigurace použití klíčů}

Nejdříve je potřeba distribuovat soubor s~veřejným klíčem na vzdálený počítač. K~tomu slouží
 např. program {\tt scp}. Každý z~uživatelů si může specifikovat sadu klíčů pro přístup k~danému
 stroji v~souboru \verb|~/.ssh/authorized_keys|. Nový klíč do tohoto souboru přidáme např. takto
 (všimněte si, že se klíč přidává na konec souboru pomocí \verb|>>|):

\begin{verbatim}
cat id_rsa.pub >> ~/.ssh/authorized_keys
\end{verbatim}

Nyní se již můžeme připojit ke vzdálenému počítači pomocí vytvořených klíčů. Pokud jsme klíč
 na lokálním počítači umístili do výchozího umístění, je klíč použit automaticky. Pokud jsme
 zvolili jiné umístění, je potřeba program {\tt ssh} informovat o~umístění klíče parametrem
 {\tt -i}, nebo volbou {\tt IdentityFile} v~konfiguračním souboru. Informace o hledaných
 klíčích jsou zobrazeny po použití parametru {\tt -v}.

