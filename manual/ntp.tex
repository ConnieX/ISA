%%%%%%%%%%%%%%%%%%%%%%%%%%%%%%%%%%%%%%
\section{Network Time Protokol}
\label{ntp}
NTP \cite{rfc1305} umožňuje synchronizovat čas mezi uzly v~síti. Tento protokol se dokáže
vypořádat s~proměnlivou dobou přenášení paketu po síti. NTP organizuje servery hierarchicky do úrovní.
Tyto úrovně se nazývají {\em stratum}. Nejnižší hodnota 0 označuje samotný zdroj přesného času (např.
GPS). Stratum 1 pak označuje servery, které jsou synchronizovány právě s~referenčním zdrojem. Stratum 2
jsou servery synchronizovány se servery stratum 1, atd. \cite{rfc5905}.

Implementaci protokolu NTP zajišťuje balík aplikací {\em ntp}. Tento balík se skládá z~několika
aplikací. V~rámci laboratorního cvičení se použijí aplikace {\tt ntpd}, {\tt ntpq} a případně {\tt ntpdc}.
Chrony\footnote{\url{https://chrony.tuxfamily.org/}} nabízí alternativní
implementaci NTP a v~bezpečnostním auditu~\cite{securing_ntp} překonal balík
{\em NTP}.

\subsection{ntpd}
NTPD je aplikace, která běží na pozadí a neustále provádí kontrolu času s~nastavenými servery a případně upravuje lokální čas. Úprava lokálního času se provádí úpravou rychlosti běhu lokálního času. Pokud je lokální čas pozadu, resp. se předbíhá, tak se {\em zrychlují}, resp. {\em zpomalují} systémové hodiny. Tento způsob úpravy času znamená, že pokud je čas odchýlen o~několik minut, bude nějakou dobu trvat, než dojde k~jeho srovnání. Na druhou stranu se tak zabrání skokové změně a navíc čas se nikdy neposune do minulosti. Pokud je lokální čas odchýlen o~více než 1000 sekund (necelých 17 minut) aplikace to vyhodnotí jako chybu a skončí. Pokud se tak stane, objeví se zpráva v~systémovém logu. Zda aplikace běží, lze zjistit například příkazem {\tt ntpq -p} (aplikace skončí s~hláškou {\em ntpq: read: Connection refused} pokud není ntpd spuštěn).

Démona je možné spustit příkazem:
\begin{verbatim}
systemctl start ntp.service
\end{verbatim}
Aby se předešlo problému v~případě, kdy např. hardwarové hodiny jdou špatně a při vypnutí může dojít k~odchýlení lokálního času o~více než 17 minut, je možné vynutit okamžité nastavení času příkazem
\begin{verbatim}
ntpd -qg
\end{verbatim}
({\tt -q} pro vynucení okamžitého nastavení času a {\tt -g} pro vypnutí kontroly kdy je rozdíl větší než 1000 sekund).

\subsubsection{Konfigurace}
Základním konfiguračním souborem je {\tt /etc/ntp.conf}. Tento konfigurační soubor obsahuje mnoho
konfiguračních voleb. Nastavení serverů NTP zajišťuje konfigurační volba {\em server}, která má jeden
parametr (adresu nebo jméno NTP serveru). Tato volba se může vyskytovat opakovaně, klient pak využívá
větší počet serverů a tím lze dosáhnout vyšší přesnosti. Volba server mimo jiné znamená, že lokální čas
se může nastavit podle uvedeného serveru, ale nemůže tomu být naopak. V~jiných případech může být volba
{\em server} nahrazena volbou {\em peer}, která umožňuje, aby se i server synchronizoval podle lokálních
hodin. Tato volba je užitečná, pokud je více ekvivalentních serverů, k~zajištění, že se budou
synchronizovat navzájem mezi sebou. Dalšími možnostmi jsou {\em broadcast} a {\em manycastclient}. Tyto
možnosti využívají broadcastového nebo skupinového vysílání (pokud je nutné synchronizovat velký počet
uzlů může tato možnost šetřit síťové zdroje). Server v~této konfiguraci vysílá na broadcastovou nebo
multicastovou adresu informace o~správném čase v~pravidelných intervalech a klienti zpracovávají tyto
informace.

Jinou užitečnou volbou je {\em restrict}, která slouží pro řízení přístupu. Aplikací {\tt ntpd} mohou být zaslány různé požadavky přes síť (viz aplikace {\tt ntpq}/{\tt ntpdc}). Je vhodné dovolit některé dotazy jen z~určitého uzlu nebo podsítě. Využití této volby může být také užitečné v~případě, že se pro nastavování času využívá broadcastu nebo multicastu a není žádoucí, aby takto vyslanou informaci klienti akceptovali z~libovolného zdroje. Základní tvar tohoto příkazu je:
\begin{verbatim}
restrict <adresa> [mask <maska>] [<jeden či více příznaků>]
\end{verbatim}

Příznak definuje omezení pro danou adresu/síť:
\begin{itemize}
  \item ignore -- zahazovat všechny pakety
  \item nomodify -- povolí pouze dotazy, požadavky měnící stav serveru jsou zahazovány
  \item noquery -- zakáže dotazy pomocí {\tt ntpq} a {\tt ntpdc}, synchronizace času není ovlivněna
  \item notrust -- zahazovat neautentizované pakety
\end{itemize}
Další příznaky a jejich popis lze nalézt v~manuálových stránkách {\tt man ntp.conf}.

Pokud budou aplikace {\tt ntpq} a {\tt ntpdc} používané i pro změnu konfigurace, pak je nezbytné nastavit autentizaci. Protokol nabízí možnost využití symetrické i asymetrické kryptografie. Pro použití symetrické kryptografie jsou k~dispozici tyto volby:
\begin{itemize}
  \item keys -- tato volba má jeden parametr, který udává název souboru, který obsahuje používané klíče (obvykle {\tt /etc/ntp.keys},
  \item trustedkey -- výčet klíčů, kterým se bude důvěřovat,
  \item requestkey -- seznam klíčů, které mohou být použity aplikací {\tt ntpdc},
  \item controlkey -- seznam klíčů, které mohou být použity aplikací {\tt ntpq}.
\end{itemize}
Formát souboru {\tt /etc/ntp.keys} má následující tvar:
\begin{verbatim}
<číslo klíče> <typ> <heslo>
\end{verbatim}
Typ může nabývat čtyř hodnot. Hodnoty {\tt S} a {\tt N} používají bitový formát a běžně se neužívají. Hodnoty {\tt A} a {\tt M} používají textový řetězce délky 1 až 8 znaků a určují způsob zašifrování při přenosu. Nejčastěji se užívá možnost {\tt M}, která značí použití DES nebo MD5. Příklad souboru pak může vypadat následovně:
\begin{verbatim}
1 M heslo1
2 M secret
3 M passwd
\end{verbatim}
Konfigurace v~{\tt /etc/ntp.conf} potom vypadá:
\begin{verbatim}
keys /etc/ntp.keys
trustedkey 1 2 3
requestkey 2
controlkey 1 3
\end{verbatim}
Toto značí, že důvěryhodné jsou všechny tři klíče. Aplikace {\tt ntpdc} se může autentizovat pouze klíčem 3 a aplikace {\tt ntpq} klíči 1 a 3.


\subsection{Aplikace ntpdc a ntpq}
Oba dva programy nabízejí v~podstatě podobné možnosti konfigurace ntp serveru. Rozdílů je mezi těmito programy několik. První, který byl již uveden výše, je v~tom, že každá z~těchto aplikací může používat jinou množinu klíčů. Podstatnější rozdíl z~uživatelského hlediska je v~tom, že aplikace {\tt ntpdc} má přesně definovaný výčet příkazů, které lze aplikovat, a proto může měnit jen to, co je v~aplikaci definováno. Naproti tomu {\tt ntpq} disponuje příkazem {\tt :config}, kterému se jako parametry předávají konfigurační volby, které se používají v~souboru {\tt /etc/ntp.conf}. Další rozdíl je ve formátu zpráv, pomocí kterých komunikují se serverem.

Obě aplikace používají pro komunikaci s~aplikaci {\tt ntpd} zasílání zpráv přes síťové rozhraní, bez ohledu na to, zda běží lokálně nebo vzdáleně. Hlavní výhoda však spočívá v~tom, že tyto aplikace dovolují měnit konfiguraci {\tt ntpd} za běhu. Možnost změny parametru přes síť může být nežádoucí a proto, je-li tato možnost povolena, je vhodné omezit pomocí volby {\em restrict} přístup pro změnu pouze přes loopback rozhraní.

Příklad výstupu volání {\tt ntpq -p}:

\begin{verbatim}
     remote           refid      st t when poll reach   delay   offset  jitter
==============================================================================
+rhino.cis.vutbr 248.205.243.78   3 u  425 1024  377   10.070   -4.280  10.575
+tik.cesnet.cz   195.113.144.238  2 u  622 1024  377    4.796   -3.464  16.528
*tak.cesnet.cz   .GPS.            1 u  364 1024  377    5.008   -3.625   6.228
\end{verbatim}

Výpis obsahuje následující hodnoty:

\begin{description}

  \item[remote] Klient se synchronizuje vůči třem serverům: rhino.cis.vutbr.cz,
    tik.cesnet.cz a tak.cesnet.cz. Hvězdičkou je označený primární zdroj času,
    plusem sekundární zdroje času.

  \item[refid] Primární zdroj času pro vzdálený server NTP.

  \item[stratum] Počet skoků od vzdáleného serveru k přesnému zdroji času (1
    znamená, že zdroj přesného času je přímo připojen, 16 znamená, že vzdálený
    server je nedosažitelný).

  \item[when] Počet sekund od posledního kontaktu se serverem.

  \item[poll] Počet sekund mezi jednotlivými dotazy protokolem NTP.

  \item[reachability] Úspěšnost posledním 8 pokusů o kontakt vzdáleného serveru
    v osmičkové soustavě. (1 znamená pouze poslední pokus uspěl, 377 znamená
    všech 8 posledních pokusů uspělo).

  \item[delay, offset, jitter] Zpoždění, posun a jitter -- charakteristiky
    vzdáleného zdroje času, podle kterého se volí primární a sekundární zdroje
    času.

\end{description}


