
\section*{Cíle laboratoře}
\begin{itemize}
  \item Základní seznámení se synchronizací času a protokolem NTP.
  \item Zabezpečený terminál SSH a připojení pomocí klíčů.
  \item Systém DNS.
  \item Zabezpečení DNSSEC.
\end{itemize}

\section*{Základní instrukce}
\begin{itemize}
  \item Přihlaste se do OS Kali Linux (F3), user/password {\tt user}/{\tt user4lab}.
  \item Otevřete si příkazovou řádku pro uživatele {\tt user}.
  \item Otevřete si příkazovou řádku pro uživatele {\tt root} příkazem {\tt su}
  (switch user).
  \item V případě potřeby si otevřete další terminál v novém okně.
  \item Pro editaci konfiguračních souborů použijte libovolný editor (např.
  nano, mcedit, vim, gedit).
  \item Pracujte ve dvojicích, zkontrolujte, že máte počítač propojen
    přímým kabelem s přístupovým přepínačem (rozhraní eth0, zdířka E na patch
    panelu).
  \item Příkazem {\tt systemctl stop NetworkManager.service} vypněte
    NetworkManager spravující rozhraní eth1 (doporučené, volitelně; zadání dále
    předpokládá, že je NetworkManager vypnutý).

\end{itemize}

\section*{Úkoly}
\section{NTP}

Vašim úkolem je zajistit synchronizaci hodin počítače. V rámci bezpečnosti je
důležité provozovat počítač se správným časem kvůli časovým razítkům
označujícím platnost a neplatnost klíčů, certifikátů, podpisů apod.

\begin{enumerate}

  \item Zkontrolujte obsah konfiguračního souboru {\tt /etc/ntp.conf}. Ujistěte
    se, že je nastavena synchronizace proti serveru {\tt ntp.fit.vutbr.cz}.

  \item Zapněte démona pro NTP ({\tt systemctl start ntp.service}).
    Sledujte postup synchronizace příkazem {\tt
    watch ntpq -p}. Nemusíte čekat na dokončení synchronizace, nechte okno
    spuštěné a pokračujte dalšími úkoly.

\end{enumerate}

\section{Vzdálený terminál -- SSH}

Vytvořte autentizační klíče pro komunikaci protokolem SSH (chráněný heslem i
nechráněný). Pracujte pod uživatelem {\tt user}. Zkontrolujte, že pracujete pod
uživatelem {\tt user} příkazem {\tt whoami}, který vypisuje jméno aktivního
uživatele.

Namísto řetězce {\tt <login>} používejte své studentské přihlašovací jméno.

\begin{enumerate}

  \item Ukončete démona spravujícího hesla k použitým klíčům SSH: \\ {\tt killall
    gnome-keyring-daemon}.

  \item {\bf Bezpečné připojení na vzdálený počítač bez autentizačních klíčů.}

    \begin{enumerate}

      \item Přihlaste se příkazem {\tt ssh {\em host}} na vzdálený počítač (např.
        sousední stanici jménem hXX, kde XX je číslo počítače, uživatel {\tt user}).

      \item Příkazem {\tt exit} nebo stiskem {\tt Ctrl+D} spojení ukončete.

    \end{enumerate}

  \item {\bf Vytvoření veřejného a privátního klíče.}

    \begin{enumerate}

      \item Příkazem {\tt ssh-keygen} vygenerujte implicitní klíč. Neměňte jeho
        název a zvolte heslo o délce alespoň pěti znaků.

      \item Příkazem \verb|ssh-keygen -N "" -f ~/.ssh/nopass -C <login>@nopass| vygenerujte autentizační klíč bez hesla.

      \item Příkazem {\tt ssh-keygen -N <heslo> -f \textasciitilde/.ssh/pass -C
        <login>@pass} vygeneruje klíč s heslem jiným než pro výchozí klíč.

      \item Ověřte obsah a přístupová práva u nově vzniklých souborů (ls -l \textasciitilde/.ssh). Jak
se liší práva mezi souborem s privátním a veřejným klíčem?

    \end{enumerate}

  \item {\bf Distribuce klíčů}

    \begin{enumerate}

      \item Všechny tři veřejné klíče si zkopírujte na vzdálený počítač
        (např. {\tt  cat \textasciitilde/.ssh/*.pub | ssh hXX 'cat >>
        .ssh/authorized\_keys'}.

      \item Přihlaste se na vzdálený počítač příkazem {\tt ssh hXX}. Jaké heslo bylo nutné zadat?

      \item Na vzdáleném počítači připojte obsah souboru, který jste zde nakopírovali na konec souboru
        {\tt \textasciitilde/.ssh/authorized\_keys} \\ (např. {\tt cat /tmp/<login> >>
        \textasciitilde/.ssh/authorized\_keys}).

      \item Odhlaste se a zkuste se znovu přihlásit. Jaké heslo bylo nyní nutné zadat? Zkuste
zadat špatné heslo a které další klíče se použily? Při experimentech můžete také
využít tzv. verbose režim ssh ({\tt ssh -v}). Pro experimenty s identitou
využijte přepínač {\tt -i}.

    \end{enumerate}

  \item {\bf Konfigurace použití klíčů}

    \begin{enumerate}

      \item Stejným způsobem nakopírujte své veřejné klíče ještě na další počítač v laboratoři.

      \item Vytvořte soubor \textasciitilde/.ssh/config pomocí svého oblíbeného editoru a
vyplňte podle vzoru (hostname definujte jen pro jeden ze dvou strojů, na který jste
kopírovali veřejné klíče):
\begin{verbatim}
Host hXX hXX.netlab.fit.vutbr.cz
 IdentityFile ~/.ssh/pass

Host *
 IdentityFile ~/.ssh/id_rsa
 IdentityFile ~/.ssh/nopass
\end{verbatim}
      \item Přihlaste se postupně na oba počítače. Které klíče se použily? Co se stane, když
pro klíč zadáte špatné heslo?
%{\em Používejte FQDN, např. h01.netlab.fit.vutbr.cz.}

    \end{enumerate}

  \item {\bf Omezení použití klíčů}

    Nyní bude naším cílem omezení použití klíče, který není chráněn heslem tak,
    aby pomocí něj bylo možné na vzdáleném serveru vykonat jen konkrétní příkaz.

    \begin{enumerate}

      \item Přihlaste se na počítač, kam jste nakopírovali své veřejné klíče a
        upravte soubor s autorizovanými veřejnými klíči tak, že na začátek
        řádku s klíčem nopass (řádek poznáte tak, že končí řetězcem
        {\tt <login>@nopass}) napíšete
        \verb|command="ntpq -p" | (od původního
        obsahu oddělený jednou mezerou).

      \item Odhlaste se ze vzdáleného počítače a znovu se na něj přihlaste
        příkazem {\tt ssh hXX -i \textasciitilde/.ssh/nopass}. Aplikovalo se
        omezené využití klíče?

    \end{enumerate}


  \item {\bf Pohodlné opakované použití klíče zabezpečeného heslem.}

    \begin{enumerate}

      \item Spusťte program {\tt gnome-keyring-daemon}.

      \item Připojte se ke vzdálenému počítači za použití klíče chráněného
        heslem.

      \item Odhlaste se a znovu přihlaste. Museli jste zadávat znovu heslo?

    \end{enumerate}

\end{enumerate}

\section{Konfigurace DNS serveru}
\begin{enumerate}

  \item Pracujte ve dvojicích. Jeden z dvojice bude konfigurovat server (PC A)
    a druhý počítač bude mít roli klienta (PC B). Propojte rozhraní eth1 (zdířka
    D na patch panelu)
    mezi počítači ve dvojici křížovým kabelem.
  \item Nastavte statické IP adresy na rozhraní eth1 vašich počítačů v souboru
    \\
    {\tt /etc/network/interfaces}:
\begin{verbatim}    
auto eth1
iface eth1 inet static
    address 192.168.X.X
    netmask 255.255.255.0
\end{verbatim}    
    Restartujte síťování {\tt systemctl restart networking.service}.
    Otestujte konektivitu příkazem {\tt ping}. Pokud konektivita nefunguje,
    ověřte nastavení IP adres {\tt ip a s} a zamyslete se, co je špatně.

  \item Upravte konfigurační soubor {\tt /etc/bind/named.conf.options} (parametr
    {\tt listen-on} sekce {\tt options}) tak, aby
    PC A naslouchalo i na adrese, která mu byla přiřazena staticky na rozhraní eth1.
    Inspirovat se můžete již uvedeným paramterem {\tt listen-on-v6}, případně v
    manuálových stránkách.
  \item Na PC A rozšiřte konfiguraci DNS serveru tak, aby obsluhoval požadavky pro vámi
    zvolenou doménu, např. {\tt isa.cz}. Doména bude obsahovat:
    \begin{itemize}
      \item Nameserver pojmenovaný {\tt server}.
      \item Záznam typu A pro {\tt server}.
      \item Záznam typu A pro jméno {\tt pcb} pro PC B.
      \item Záznam typu CNAME pro jméno {\tt pca} ukazující na {\tt server}.
    \end{itemize}
    Inzerujte platnost vámi zasílaných záznamů 1 hodina.
    Jako základ zónového souboru pro vaší
    doménu můžete použít soubor {\tt /etc/bind/db.local}, či {\tt
    /root/isa3/fit.vutbr.cz}.
    Needitujte však soubor přímo, ale zkopírujte jej do {\tt /var/cache/bind}, zóny zaregistrujte v
    souboru {\tt named.conf.local} (potřebné informace jsou k dispozici v
    laboratorním manuálu).
  \item V případě zájmu nakonfigurujte pro doménu překlad na adresy IPv6
    (záznamy AAAA).
  \item Nakonfigurujte reverzní překlad. Jako základ
    zónového souboru s reverzní zónou můžete použít např. {\tt /etc/bind/db.127}.
    Zkopírujte soubor do {\tt /var/cache/bind}, upravte jej a přidejte do
    souboru {\tt named.conf.local}.
  \item Spusťte DNS server příkazem {\tt systemctl start bind9.service}.
    Ujistěte se, že v souboru {\tt /var/log/syslog} nejsou žádné chyby.
  \item V souboru {\tt /etc/dhcp/dhclient.conf} nastavte na obou počítačích použití vlastního
    serveru DNS běžícího na PC A, na konec souboru přidejte řádek
    {\tt prepend domain-name-servers
    192.168.X.X;}. Na obou počítačích restartujte síťování: {\tt systemctl restart networking.service}.
    Ověřte nastavení IP adres a obsah souboru {\tt /etc/resolv.conf}.
  \item Ověřte, že PC A používá pro překlad lokální DNS server ({\tt dig
    localhost}). Adresu dotazovaného serveru udává položka {\em SERVER} v
    poslední části odpovědi.
  \item Ověřte, že PC B používá pro překlad lokální DNS server a dostává od něj odpověď.
  \item Ověřte zda lze z obou počítačů přeložit dopředné i zpětné záznamy.
  \item Odchyťte si komunikaci v rámci DNS na obou počítačích programem
    {\tt wireshark}. Analyzujte si jednotlivé položky a porovnejte je s výpisem
    programu {\tt dig}.

\end{enumerate}

\section{Zabezpečení DNS -- DNSSEC}

\begin{enumerate}
  \item Na PC B spusťte prohlížeč IceWeasel. Zobrazte si webové prezentace v
    doménách {\tt www.fit.vutbr.cz} a {www.seznam.cz}. Všimněte si klíče v pravé
    části adresního řádku.

  \item Na PC A vytvořte zónové soubory pro domény {\tt fit.vutbr.cz} a {\tt
    seznam.cz}. Nastavte IP adresu počítače {\tt www} v obou doménách na adresu
    PC A. Zónové soubory najdete připraveny ve složce {\tt /root/isa3}. Zónové
    soubory přidejte do souboru {\tt /etc/bind/named.conf.local} a restartujte
    Bind.

  \item Na PC A se v příkazové řádce přesuňte do adresáře {\tt /root/isa3}.
    Spusťte zde jednoduchý webový server příkazem {\tt python -m
    SimpleHTTPServer 80}.

  \item Na PC B ověřte, že se doménová jména {\tt www.seznam.cz} a {\tt
    www.fit.vutbr.cz} překládají na IP adresu PC A.

  \item Na PC B si zobrazte webové prezentace v doménách {\tt www.fit.vutbr.cz}
    a {www.seznam.cz}. Všimněte si klíče v pravé části adresního řádku.

  \item Volitelně zabezpečte DNS server pomocí DNSSEC. Postup hledejte v
    laboratorním manuálu.

\end{enumerate}

\section{Ukončení práce v laboratoři}
\begin{itemize}
  \item Počítač vypněte dávkou {\tt /root/isa3/clean}.
\end{itemize}
