\section*{Cíle cvičení}
\begin{itemize}
	\item Seznámení se se základní prací v~OS Linux.
	\item Seznámení se se základními nástroji pro zjišťování konfigurace zařízení.
	\item Analýza síťového provozu pomocí Wireshark.
	\item Seznámit se s manuální a dynamickou konfigurací IPv4 a IPv6 na OS Linux
\end{itemize}

\section*{Pokyny}
\begin{itemize}
\item Do zadání nepište, slouží pro další skupiny. K zápisu používejte pouze
  záznamový arch, který vám zůstane. Cvičící může použít zápisky v záznamovém
    archu při hodnocení cvičení.
\item Pro práci v laboratoři budeme používat OS Linux, při bootu volba F3.
\item Uživatelé a hesla pro přihlášení: \texttt{user - user4lab}, \texttt{root - root4lab}.
\item Přihlaste se jako uživatel \texttt{user}. Veškeré potřebné příkazy následně spouštějte jako \texttt{root}.
\end{itemize}

\subsection*{Příprava laboratoře}
Před samotnou konfigurací je potřeba přepojit počítače kabelem přes patch panel
pro kabely DXX (XX je číslo vašeho počítače) vepředu laboratoře. Na PC budete
pracovat s rozhraníma {\bf eth1}.

\section{Zjišťování konfigurace}
V případě, že se v OS Linux úplně neorientujete, přečtěte si kapitolu \ref{basics} v
laboratorním manuálu --- Základní konfigurace linuxového serveru. V~této části cvičení se
budeme zabývat převážně zjišťováním síťové konfigurace systému. Veškeré potřebné
informace, které budete potřebovat ke splnění této části cvičení, naleznete
v~sekci \ref{basic_ipconfig} laboratorního manuálu --- Konfigurace síťových zařízení. V~případě, že si nejste jistí některým příkazem, neváhejte nahlédnout do manuálové stránky.

\begin{enumerate}
\item Vypište konfiguraci vašeho stroje (MAC adresu, IPv4 adresu, masku, síť, broadcastovou adresu).
\item Zobrazte si záznamy v~routovací a ARP tabulce. Zapište pravidlo týkající se defaultní cesty. Následně k~IPv4 ve vypsaném záznamu přiřaďte MAC adresu.
\item Otestujte konektivitu k~výchozí bráně a následně konektivitu do internetu.
\item Vypište implicitní servery DNS a soubor, ve kterém jste tuto informaci nalezli.
\item Upravte patřičný soubor tak, aby po spuštění příkazu \texttt{ping google},
  byl ping proveden vůči IPv4 adrese 8.8.8.8. Zapište jak a který soubor jste
    upravili a jaký záznam jste přidali.
\item Vypište aktivní TCP spojení, vyberte jeden záznam, zapište si ho a popište význam jednotlivých položek. Pokud se žádné TCP spojení nezobrazuje, nějaké vygenerujte, například pomocí webového prohlížeče.
\item Zobrazte systémové události pomocí programu \texttt{journalctl}.
\item Zobrazte pouze události týkající se NetworkManager.
\item Pokuste se jako uživatel user spustit Wireshark s pomocí programu \texttt{sudo}. Následně nalezněte v~logu zprávu, která byla zaznamenána v~případě neúspěšného přihlášení.
\end{enumerate}

\section{Wireshark}
V~této části cvičení se budeme zabývat analýzou a zachytáváním provozu
v~programu Wireshark. Spuštění Wireshark provedete příkazem \texttt{wireshark}
pod uživatelem \texttt{root}. Veškeré potřebné informace, které budete
potřebovat k~této části cvičení, naleznete v~kapitole \ref{wireshark} laboratorního manuálu
--- Analýza síťového provozu programem Wireshark.

\begin{enumerate}
  \item Pomocí programu Wireshark začněte \textbf{zachytávat} pouze HTTP komunikaci na výchozím portě
    (výchozí porty je možné nalézt např. v \texttt{/etc/services}). Zapište
    použitý \emph{capture filter} do odpovědního archu. Spusťte si prohlížeč a
    načtěte stránku \texttt{http://www.fit.vutbr.cz} (po zapnutí zachytávání
    provozu). Zachycený provoz uložte do adresáře \texttt{/tmp}. (přípona \texttt{.pcap})
\item Vypište zdrojovou a cílovou IPv4 adresu a MAC adresu odeslaného a
  přijatého paketu. Zamyslete se nad tím, co vypsané MAC adresy a IPv4 adresy
    identifikují --- nalezené identifikátory porovnejte s identifikátory
    vypsanými v předchozích bodech zadání.
\item Zahajte znovu zachytávání komunikace, nyní bez použití filtru pro HTTP. V příkazové řádce odstraňte ARP záznamy (příkaz \texttt{ip neighbor ...}). Ve wiresharku zobrazte veškerou komunikaci, následně vyfiltrujte pouze ARP a ICMP pakety. Vygenerujte ICMP komunikaci. Analyzujte obsah ARP paketů. Zapište, co jste zadali do filtru.
\item Zachyťte pouze HTTP a DNS provoz (na výchozích portech). Ve webovém prohlížeči zkuste otevřít několik stránek na různých URL adresách. Analyzujte obsah a posloupnost DNS paketů a následných HTTP paketů.
\item Znovu si otevřete dříve uložený soubor s nešifrovanou komunikaci pomocí
  HTTP, zobrazte si TCP stream této komunikace (\emph{Follow TCP stream}).
    Najděte dotaz, který odpovídá stránce zobrazené v prohlížeči. Pokuste se
    zachytit šifrovanou komunikaci HTTPS na libovolnou stránku, analyzujte
    zachycenou komunikaci, zaměřte se také na provoz DNS; využijte funkci \emph{Follow TCP stream}.
\item Do odpovědního archu slovně popište význam funkce \emph{Follow TCP stream},
    zamyslete se nad formátem zobrazených dat funkcí \emph{Follow TCP stream} a
    rozdílem oproti výchozímu pohledu na data ve formě paketů.
\end{enumerate}

\section{Konfigurace IPv4 a IPv6}

\subsection{Příprava laboratoře}
Jako první krok je potřeba vypnout službu \texttt{NetworkManager.service}.
Ta se stará o správu síťových konfiguračních profilů na vyšší úrovni a pokoušela by se měnit naši konfiguraci.
Informace o správě systémových služeb najdete v sekci \ref{sluzby} laboratorního manuálu.
Po vypnutí služby odstraňte konfiguraci z rozhraní \textbf{eth1} pomocou příkazu \texttt{ip}.
Možnosti manuální konfigurace IP adres jsou v sekci \ref{ipconfig} --- Konfigurace síťování koncových zařízení.

\subsection{Manuální konfigurace IPv4}

Teorie \emph{IPv4 adresování} je popsaná v sekci \ref{adresy_ipv4} laboratorního manuálu --- Adresy v IPv4 síti.

\begin{enumerate}
    \item Zvolte nejdelší možnou masku sítě \texttt{192.168.0.0} tak, aby síť obsahovala prostor pro 100 koncových stanic.
    \item Z takto vytvořené sítě si vyberte 2 vhodné adresy, které nakonfigurujte na 2 počítače ve Vaší síti.
    \item Správnou konfiguraci si ověřte příkazem {\tt ping}.
\end{enumerate}

\subsection{Dynamická konfigurace IPv4}

Vaší úlohou bude na jednom počítači nakonfigurovat DHCP server.
Na druhém počítači spustíte DHCP klienta, který si od nakonfigurovaného serveru požádá o přidělení vhodné IPv4 adresy.
Teorie konfigurace ISC DHCP serveru a použití DHCP klienta je popsaná v sekci \ref{dhcp} --- Dynamická konfigurace IPv4 - DHCP.

\begin{enumerate}
    \item Počítač, na kterém běží server musí být nakonfigurovaný manuálně tak, že na síťovém rozhraní má staticky nakonfigurovanou IP adresu ze správné sítě.
      Zároveň tato adresa nesmí být v rozsahu adres, které poskytuje služba DHCP.
    \item Nakonfigurujte DHCP server tak, aby poskytoval IPv4 adresy pro stejnou síť jakou jste si zvolili pro manuální konfiguraci.
    \item Nakonfigurujte DHCP server tak, aby šířil informaci o DNS serveru na adrese {\bf 10.10.10.1}.
    \item Povolte automatické spuštění služby ISC DHCP serveru.
    \item Na druhém počítači smažte manuální IP konfiguraci a spusťte DHCP klientskou aplikaci.
    \item Správnou konfiguraci si ověřte příkazem {\tt ping}.
    \item Správnou konfiguraci DNS serveru ověřte v konfiguračním souboru (\texttt{resolv.conf}).
\end{enumerate}

\subsection{Manuální konfigurace IPv6}
Teorie \emph{IPv6 adresování} je popsaná v sekci \ref{adresy_ipv6} --- Adresy v IPv6 síti.
Možnosti manuální konfigurace IP adres jsou stejné jako pro IPv4.

\begin{enumerate}
    \item Zvolte si adresu sítě vhodnou pro použití v privátních lokálních
        sítích, s prefixem délky 64 bitů. Prvních 48 bitů zvolte podle popisu v
        sekci \ref{adresy_ipv6}, pro vygenerování unikátního Global ID můžete
        použít web \url{https://cd34.com/rfc4193/}, zbývajících 16 bitů (Subnet
        ID) si můžete zvolit libovolně.
    \item Pro zajímavost si můžete unikátnost vygenerovaného Global ID
        zkontrolovat na \url{https://www.sixxs.net/tools/grh/ula/list}.
    \item Z takto vytvořené sítě si vyberte 2 vhodné adresy, které
        nakonfigurujte na 2 počítače ve Vaší síti.
    \item Správnou konfiguraci si ověřte příkazem {\tt ping6}.
\end{enumerate}

\subsection{Dynamická konfigurace IPv6}
Stejně jako u dynamické IPv4 konfigurace bude každý počítač ve dvojici plnit
jinou roli. Jeden počítač bude v roli směrovače vysílat Router Advertisement
zprávy informující o konfiguraci sítě. Druhý počítač bude v roli klientského
zařízení, které si po připojení do sítě na základě RA zprávy nakonfiguruje
správnou IPv6 adresu.

Dynamická IPv6 konfigurace je popsaná v sekci \ref{dynipv6} .

\begin{enumerate}
    \item Na obou počítačích si ponechte IP adresy nakonfigurované během
        předcházejících úkolů. Pokud došlo ke smazání IPv6 adresy, použijte příkaz \texttt{ip a a fe80::X/64 dev eth1}, kde písmeno X značí číslo PC.
    \item Na jednom z počítačů nakonfigurujte službu \texttt{radvd.service} tak,
        aby ve Vaší síti (rozhraní eth1) šířila informace o prefixu sítě, který
        jste si vygenerovali pro manuální konfiguraci.
    \item Na stejném počítači povolte směrování IPv6 provozu pomocí příkazu\\
        \texttt{sysctl net.ipv6.conf.all.forwarding=1}\\
        Tím Váš počítač začne plnit roli směrovače pro IPv6 provoz a bude moci
        šířit RA zprávy.
    \item Povolte automatické spuštění služby \texttt{radvd.service} po startu systému.
    \item Potvrďte správnost nově nakonfigurovaných adres (příkaz \texttt{ip
        address}) a funkčnost sítě pomocí \texttt{ping6}.
    \item Pro zajímavost se můžete podívat na obsah RA zpráv pomocí aplikace
        \texttt{radvdump}, nebo \texttt{wireshark} (filter \texttt{icmpv6.type
        == 134}).
\end{enumerate}

\section{Ukončení práce v~laboratoři}
Jakmile máte veškerou práci hotovou, ohlaste se u vyučujícího, který Vám
konfiguraci zkontroluje. Pak můžete spustit skript {\tt /root/isa1/clean} (jako
\texttt{root}).
